\section{Objetivos}

\subsection{Objetivo General}

Garantizar recomendaciones seguras y confiables para nómadas digitales que visiten Bogotá, mediante el análisis avanzado de reseñas de usuarios, factores urbanos, geográficos y culturales. La aplicación priorizará la seguridad del usuario al evaluar riesgos en cada destino, considerando niveles de criminalidad, accesibilidad, iluminación y afluencia de personas. Además, ofrecerá opciones que equilibren protección, comodidad y preferencia personal, permitiendo una experiencia turística segura y adaptada a sus necesidades.

\subsection{Objetivos Específicos}
\begin{itemize}
    \item  Desarrollar un sistema de análisis de reseñas de usuarios que permita identificar las preferencias y experiencias previas de los nómadas digitales en diferentes puntos turísticos de la ciudad de Bogotá. 
    \item  Integrar datos urbanos, geográficos y culturales para evaluar y sugerir zonas o actividades que ofrezcan un equilibrio óptimo entre seguridad, accesibilidad, y autenticidad cultural para los usuarios. 
    \item  Crear un perfil personalizado de cada usuario que permita recomendar actividades, zonas o servicios basados en sus preferencias individuales, como el tipo de alojamiento, transporte, espacios de trabajo, y atracciones turísticas.
    \item Implementar un sistema de alerta y recomendación basado en la seguridad que informe a los usuarios sobre las zonas seguras y confortables para trabajar, hospedarse, y disfrutar de su tiempo libre en Bogotá. 
    \item Optimizar la experiencia del usuario mediante la inclusión de sugerencias en tiempo real basadas en cambios en el entorno urbano, como clima, eventos culturales, o cambios en las reseñas más recientes de los puntos turísticos. 
    \item Monitorear y ajustar el algoritmo de recomendación para mejorar la precisión de las sugerencias, ajustándose continuamente a las nuevas reseñas de usuarios y datos dinámicos de la ciudad.

\end{itemize}