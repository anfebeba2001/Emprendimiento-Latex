
\subsection{Estructura financiera}
Esta sección tiene como finalidad proporcionar información fundamental del plan de negocio en la cual se refleja la viabilidad del proyecto fundamentada en el estudio financiero del proyecto. Esta se basa en 6 pilares.
\begin{itemize}
    \item Estado de resultado pro forma proyectado a tres años:Es una proyección del comportamiento financiero de la empresa en el futuro,estimando ingresos,costos,gastos y utilidades basados en supuestos realistas.Esto con la finalidad de prever rentabilidad y planificar estrategias operativas o de inversión.
    \item Balance general proyectado a cinco años:Refleja la situación financiera de la empresa(activo,pasivos y patrimonios)a cinco años.Los activos siendo efectivo,inventario y propiedades.Los pasivos corresponden a deudas a corto/largo plazo.El patrimonio comprende capital social y utilidades acumuladas.Este balance permite evaluar la solvencia y la capacidad de crecimiento de la empresa a futuro.
    \item Flujo de caja pro-forma proyectado a cinco años:Se realiza un pronostico de entradas y salidas de efectivo ya sea en operaciones financieras,inversiones o financiamientos durante cinco años, esto con la finalidad de identificar necesidades de liquidez, capacidad para pagar deudas y evitar crisis de efectivo.
    \item Análisis del punto de equilibrio: Es el cálculo del volumen de ventas necesario para cubrir costos totales(fijos y variables) donde las utilidades son cero.Esto con la finalidad de determinar la viabilidad de un negocio y fijar metas mínimas de ventas.
\end{itemize} 


