{\color{red}
\subsection{Estructura financiera}

Este pilar es el núcleo del plan de negocio, proporcionando información vital sobre la
viabilidad del proyecto. Integra las finanzas necesarias para llevar a cabo el proyecto y se basa en 6 puntos fundamentales.

\begin{itemize}
    \item Estados de resultados pro-forma proyectados a tres años: Ofrecen una visión del comportamiento del futuro del negocio, considerando su estructura financiera. Se calculan tomando en cuenta las  siguientes variables: número de unidades vendidas y su precio, costo de ventas por unidad, costos fijos, costos variables, intereses (de créditos) e impuestos. El resultado será la utilidad neta. Estos estados se establecerán a través de este análisis.

    \item Balance general proyectado a cinco años: Es un esquema estructurado de dos variables: los componentes de la empresa y su financiación. Incluye el mobiliario y equipo (activos de la compañía), así como los recursos utilizados para adquirirlos. 

    \item Flujo de caja pro-forma proyectado a cinco años: En este punto, se definen las políticas de cuentas por cobrar y el ciclo de ventas. El objetivo es que este reporte responda a las siguientes preguntas: ¿Cuándo se solicitará capital? y ¿de dónde se obtendrá este capital?.

    \item Análisis del punto de equilibrio: Es el número de unidades de productos o servicios que la empresa debe vender para cubrir los costos fijos de operación. Este dato es crucial para determinar cuándo las ventas comenzarán a generar ganancias y utilidades para la   compañía.
\end{itemize}
}