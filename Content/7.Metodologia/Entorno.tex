

{\color{blue}\subsection{Entorno}


En este pilar se consolida la organización en el mercado, realizando el análisis del entorno, integrado por la competencia, aliados y los potenciales clientes.

\begin{itemize}
    \item Matriz DOFA: Metodología diseñada para proporcionar información sobre la situación de la empresa en el mercado, identificando cuatro aspectos clave: Debilidades, Oportunidades, Fortalezas y Amenazas.
    \item Descripción del publico objetivo: Ofrece la presentación y análisis de la población destinataria de la propuesta de valor de la empresa, utilizando estadísticas pertinentes para centralizar la información.
    \item Investigación demográfica del mercado: Se examina el crecimiento del mercado desde un periodo específico hasta el presente, considerando el estado actual del mercado y proyectando su desarrollo a corto, mediano y largo plazo.
    \item Frecuencia de adquisición del producto: Establece un método para calcular un índice que determine la frecuencia esperada con la que los clientes potenciales puedan adquirir el producto ofrecido, obteniendo proyecciones con un margen de error mínimo.
    \item Estudio de la competencia: Basado en el benchmarking, se analizan las ideas y pilares de cada entidad competidora, centrándose en el valor agregado y las estrategias de mercadotecnia utilizadas.
\end{itemize}
}