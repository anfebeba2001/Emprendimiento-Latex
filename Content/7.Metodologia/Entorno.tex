  \subsection{Entorno}
    Para este apartado se declara la organización en el mercado, analizando el entorno que la rodea economicamente conformado por la competencia, aliados y los potenciales clientes.
    \begin{itemize}
        \item Matriz DOFA:  La Matriz DOFA es una herramienta que permite identificar las Debilidades, Oportunidades, Fortalezas y Amenazas de una empresa en el mercado. Esta metodología proporciona información valiosa sobre la situación actual de la empresa y su posición competitiva.
        \item Descripción del público objetivo: En este apartado se presenta y analiza la población destinataria de la propuesta de valor de la empresa, utilizando estadísticas pertinentes para centralizar la información.
        \item Investigación demográfica del mercado: Hay que examinar el crecimiento del mercado desde un periodo preio hasta el presente, considerando el estado actual del mercado y proyectando su desarrollo a corto, mediano y largo plazo.
        \item Frecuencia de adquisición del producto: Se propone un estimado de la cantidad de veces que se puede obtener el producto (o renovación del servicio) en un periodo determinado, considerando la cantidad de clientes potenciales y el tiempo de adquisición.
        \item Estudio de la competencia: Es necesario analizar la competencia en el mercado, sus ideas y pilares, como podrían presentar una oportunidad o una amenaza para la empresa.
    \end{itemize}
