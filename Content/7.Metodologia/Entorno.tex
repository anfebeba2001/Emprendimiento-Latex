
\subsection{Entorno}
    Para este apartado se declara la organización en el mercado, analizando el entorno que la rodea economicamente conformado por la competencia, aliados y los potenciales clientes.
    \begin{itemize}
        \item Matriz DOFA: Herramienta estratégica cuyo propósito es identificar factores internos(Fortalezas y Debilidades) y externos (Oportunidades y Amenazas) con la finalidad de tomar decisiones estrategicas.
        \item Descripción del público objetivo:Presenta el perfil detallado del cliente ideal al que se dirige el servicio, incluyendo datos demograficos (edad,género,ubicación),psicográficos(intereses,valores)y comportamiento(hábitos de compra).
        \item Investigación demográfica del mercado:Estudio de datos estadisticos de una población(edad,ingresos,educación,ocupación,etc.)para segmentar el mercado, lo cual permite observar quiénes son los posibles compradores y adaptar estrategias de marketing.
        \item Frecuencia de adquisición del producto: indica la regularidad con la que los clientes potenciales podrán consumir el servicio o producto, esto ayuda a planear estrategias de venta y fidelización.
        \item Estudio de la competencia :Realizando un análisis de los competidores directos e indirectos en el mercado,enfocandose en aspectos como precios,calidad,estrategias de marketing,participación de mercado y ventajas competitivas, se identifican oportunidadesy diferenciadores que le permiten a la empresa posicionarse mejor.
         
    \end{itemize}