\section{Alcances y limitaciones}

\subsection{Alcances}
\begin{itemize}
    \item  Servir como asistente digital para la recomendación de empresas de turismo en consumidores de estos emprendimientos analizando tendencias y experiencias de otros usuarios. 

\item Organizar un sistema de reseñas que donde se regule la transparencia y validez de cada reseña, dándole peso y respaldo a cada una con la inclusión de evidencia fotográfica y de video acerca de los servicios prestados. 

\item Promover la visualización de emprendimientos enfocados hacia la exploración de la cultura en territorio bogotano (Aerolíneas con vuelos a Bogotá, museos, casas de cultura, restaurantes... etc.) 



\end{itemize}


\subsection{Limitaciones }
\begin{itemize}
    \item  Tanto las empresas enfocadas al turismo como sus consumidores están limitados únicamente a territorio bogotano.
    \item Los estudios de mercado son reconocidos a la fecha de presentación de este proyecto ya que son sujetos a posibles fluctuaciones de la moneda nacional y su mercado. 
    \item Este proyecto solo reconocerá emprendimientos que se registren en la aplicación, no todas las empresas de territorio bogotano indiscriminadamente. 
    \item Para registrarse como una empresa proveedora de servicio de turismo debe estar directamente relacionada con la exploración cultural del territorio bogotano, esto acoge a emprendimientos como restaurantes, hoteles y aerolíneas por su estrecha relación con el desarrollo cultural, sin embargo, empresas que sean derivadas de esta exploración cultural no directamente relacionadas no estarán contempladas, tales como emprendimientos de venta de recuerdos, venta de inmuebles o vehículos. 
    \item El proyecto no propone un intermediario entre los consumidores turisticos y las empresas proveedoras de turismo, esto implica que no se recibirá dinero de los consumidores para comprar, organizar o reservar directamente en la empresa de turismo ya que se escapa de las funciones de asistente digital. 
    \item Pese a que se reconoce que la publicidad paga de empresas proveedoras de turismo impulsa su visibilidad frente a los posibles consumidores de sus servicios, esta visibilidad mantiene la filosofía transparente del proyecto y es en su mayoría meritocrática, esto implica que a pesar de que una empresa pague por aumentar su visibilidad, esto no implicará que se muestre por encima de empresas que tengan mejor calidad en la prestacion de sus servicios o productos. Se entiende entonces que esta publicidad paga es un impulso a su visibilidad mas no una garantía de ser los primeros en ser mostrados cuando se presenten opciones a los consumidores turísticos. 





\end{itemize}

