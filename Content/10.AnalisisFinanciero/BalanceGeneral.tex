\subsection{Balance General}

El balance general, junto con los estados de ganancias y pérdidas, los cambios en el patrimonio neto y los flujos de efectivo, conforman los estados financieros básicos. Su propósito principal es proporcionar información sobre la situación y el desempeño financiero, así como sobre los flujos de efectivo, de manera que sea útil para una amplia variedad de usuarios al tomar sus decisiones económicas.

La responsabilidad de la preparación y presentación de los estados financieros recae en la administración de la empresa.

En esta ocasión, abordaremos el Balance General, el cual proporciona información sobre la situación financiera de una empresa al final de un periodo contable.

La información presentada en un balance general se clasifica de tal manera que los usuarios pueden obtener datos sobre la liquidez, la fecha de vencimiento de los pasivos, la cantidad de activos asignados a inmuebles, maquinarias y equipos, así como la proporción de activos financiados por los acreedores y por los propietarios.

La información sobre la liquidez de los activos se obtiene al distinguir entre activos corrientes y no corrientes. Los activos corrientes se componen de efectivo y de los recursos que se espera convertir en efectivo a través de su venta o consumo dentro de un año o en el ciclo normal de operaciones, que puede ser más largo. El ciclo normal de operaciones es el periodo de tiempo que abarca desde la compra de inventarios, el procesamiento de esos inventarios para poder venderlos, la venta de los bienes y el cobro resultante de dichas ventas.

La situación financiera se representa mediante una serie de recursos que la empresa puede utilizar, denominados activos, y las demandas sobre esos recursos, que están representadas por los pasivos y el patrimonio neto.

\begin{center}
    Patrimonio = Activos - Pasivos
\end{center}

En la tabla \ref{balanceGeneral} se detallan los activos, pasivos y el patrimonio de cada año, comenzando desde el año 0 con una inversión inicial que inicia los procesos del negocio. Esta proyección considera las ventas desde el año contable 1 hasta el año 5.

El balance debe ser presentado para la aprobación de la Asamblea General de Accionistas por parte del Representante Legal, junto con los demás documentos a los que se refiere el artículo 446 del Código de Comercio. Dentro del plazo establecido por la ley, el Representante Legal enviará a la Superintendencia, si es el caso, una copia del balance y de los anexos que lo expliquen y justifiquen, junto con el acta en la que hayan sido discutidos y aprobados.

\vspace{2mm}
\begin{minipage}{0.8\textwidth}
\centering
\captionof{table}[{Balance General}]{ Balance General }
\label{balanceGeneral}
\includegraphics[width=1.2\textwidth]{Images/balanceGeneral.png}
\fnote{Nota. \textup{Fuente : Autores}}
\end{minipage}
\newpage