\subsection{Punto de equilibrio}

Calcular el punto de equilibrio es fundamental, ya que le indica a la empresa cuántas veces necesita cubrir sus costos fijos operativos. Este cálculo es vital para determinar cuándo las ventas comenzarán a generar ganancias. En la tabla \ref{costosFijosVariables} se muestran las proyecciones de los costos fijos y variables vinculados a las ventas del primer año.

\vspace{2mm}
\begin{minipage}{0.9\textwidth}
\centering
\captionof{table}[{Costos fijos y variables}]{ Costos fijos y variables. }
\label{costosFijosVariables}
\includegraphics[width=0.9\textwidth]{Images/costosFijosVariables.png}
\fnote{Nota. \textup{Fuente : Autores}}
\end{minipage}

El costo fijo por unidad se calcula sumando los costos administrativos y de ventas, lo que permite obtener el valor total del costo por unidad. Se proyecta un margen de valor agregado del 19\% (\$8.020), estableciendo así un precio de venta de \$50.228. Para recuperar la inversión inicial, se requiere alcanzar 1.140 ventas anuales, y a partir de ese punto, todas las ventas generarán utilidad.

\vspace{2mm}
\begin{minipage}{0.9\textwidth}
\centering
\captionof{table}[{Estimaciones punto de equilibrio}]{ Estimaciones punto de equilibrio. }
\label{calculosPuntoEquilirbio}
\includegraphics[\textwidth]{Images/estimacion.png}
\fnote{Nota. \textup{Fuente : Autores}}
\end{minipage}

En la gráfica se observa el punto de equilibrio con rectas de costos totales y ventas. En el eje X se encuentran las unidades vendidas y en el eje Y está la cantidad de dinero, de forma que con 958 ventas se igualan ambas rectas resultando el punto de equilibrio.

\vspace{2mm}
\begin{minipage}{0.9\textwidth}
\centering
\captionof{figure}[{Gráfica punto de equilibrio. }]{Gráfica punto de equilibrio. }
\label{graficaEquilibrio}
\includegraphics[width=0.9\textwidth]{Images/Punto de equilibrio.png}
\fnote{Nota. \textup{Fuente : Autores}}
\end{minipage}

Considerando los resultados presentados en la tabla \ref{puntoEquilibrio}, se resume a continuación la información requerida para determinar el punto de equilibrio.

\vspace{2mm}
\begin{minipage}{0.9\textwidth}
\centering
\captionof{table}[{Punto de equilibrio}]{ Punto de equilibrio. }
\label{puntoEquilibrio}
\includegraphics[\textwidth]{Images/tablaPuntoEquilibrio.png}
\fnote{Nota. \textup{Fuente : Autores}}
\end{minipage}