\subsection{Amortización y depreciación}

\subsubsection{Préstamo}

Con el fin de dar inicio al proyecto, se obtuvo un préstamo de una entidad bancaria. Este préstamo se adicionó al capital social de la empresa. La tabla \ref{prestamo} muestra cómo se invirtió este capital en la adquisición de los activos fijos necesarios para el proyecto, así como las condiciones bajo las cuales se otorgó el préstamo.

\vspace{2mm}
\begin{minipage}{0.9\textwidth}
\centering
\captionof{table}[{Préstamo}]{ Préstamo }
\label{prestamo}
\includegraphics[width=0.5\textwidth]{Images/prestamo.png}
\fnote{Nota. \textup{Fuente : Autores}}
\end{minipage}

\subsubsection{Amortización}


Se presenta un resumen del pago anual realizado, donde la tabla \ref{amortizacion} muestra la amortización total de la deuda, programada para ser liquidada en un periodo de 4 años. En esta tabla se detalla cómo se irá pagando la deuda dentro del plazo establecido, incluyendo el saldo final en cada periodo, hasta alcanzar un saldo de \$0, considerando los intereses correspondientes.

\vspace{2mm}
\begin{minipage}{0.9\textwidth}
\centering
\captionof{table}[{Amortización}]{ Amortización }
\label{amortizacion}
\includegraphics[width=0.9\textwidth]{Images/amortizacion.png}
\fnote{Nota. \textup{Fuente : Autores}}
\end{minipage}


\subsubsection{Depreciación}

En la tabla \ref{depreciacion} se incluyen los equipos tecnológicos que serán utilizados, junto con su depreciación derivada del uso y la obsolescencia de los dispositivos. La depreciación se estima en un 20\% anual, basada en una vida útil de 5 años.


\vspace{2mm}
\begin{minipage}{0.9\textwidth}
\centering
\captionof{table}[{Depreciación}]{ Depreciación }
\label{depreciacion}
\includegraphics[width=0.9\textwidth]{Images/depreciacion.png}
\fnote{Nota. \textup{Fuente : Autores}}
\end{minipage}