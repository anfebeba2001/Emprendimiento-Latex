\subsection{Establecimiento}

A continuación, se presenta un análisis de los costos e inversiones necesarios para la creación de la empresa, considerando aspectos como el diseño, desarrollo e implementación del software. El análisis cubre los recursos económicos, humanos, administrativos y pre-operativos requeridos para el inicio de las operaciones, detallados de la siguiente manera: 

\subsubsection{Activos fijos tangibles}

Los activos fijos tangibles se dividen en dos categorías principales: el área operacional y el área administrativa, ambas cruciales para el inicio del proyecto. La tabla \ref{activosTangiles} ofrece una descripción detallada de cada una de estas:

\vspace{2mm}
\begin{minipage}{0.9\textwidth}
\centering
\captionof{table}[{Inversión de activos fijos tangibles.}]{ Inversión de activos fijos tangibles. }
\label{activosTangiles}
\includegraphics[width=0.7\textwidth]{Images/fijosTangibles.png}
\fnote{Nota. \textup{Fuente : Autores}}
\end{minipage}

\subsubsection{Activos fijos intangibles}

Los activos fijos intangibles se refieren a los procesos financieros y legales que la empresa debe considerar, descritos en la tabla \ref{activosIntangibles}.

\vspace{2mm}
\begin{minipage}{0.9\textwidth}
\centering
\captionof{table}[{ Inversión de activos fijos intangibles.}]{ Inversión de activos fijos intangibles. }
\label{activosIntangibles}
\includegraphics[width=0.7\textwidth]{Images/fijosIntangibles.png}
\fnote{Nota. \textup{Fuente : Autores}}
\end{minipage}

\subsubsection{Total activos fijos}
La tabla \ref{totalActivos} presenta un resumen del total de inversión necesaria para la adquisición de los activos fijos y el capital de trabajo requerido la ejecución del proyecto.

\vspace{2mm}
\begin{minipage}{0.9\textwidth}
\centering
\captionof{table}[{ Total activos fijos.}]{ Total activos fijos. }
\label{totalActivos}
\includegraphics[width=0.7\textwidth]{Images/totalActivos.png}
\fnote{Nota. \textup{Fuente : Autores}}
\end{minipage}

\subsubsection{Financiamiento}
En la tabla \ref{financiacionn} se muestra la inversión requerida para la fase inicial del proyecto, la cual debe evaluarse considerando el aporte de capital social y el financiamiento por parte de una entidad.

\vspace{2mm}
\begin{minipage}{0.9\textwidth}
\centering
\captionof{table}[{Financiación}]{ Financiación. }
\label{financiacionn}
\includegraphics[width=0.7\textwidth]{Images/financiacion.png}
\fnote{Nota. \textup{Fuente : Autores}}
\end{minipage}