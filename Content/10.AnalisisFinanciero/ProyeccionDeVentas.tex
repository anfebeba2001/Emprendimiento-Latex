\subsection{Proyección de ventas}

Con base en el apartado anterior, donde se abordaron las ventas y los objetivos de recaudo a nivel individual, en este apartado se realiza un análisis e identificación con el propósito de proyectar los resultados financieros futuros de la empresa. Este análisis permitirá observar cómo el producto se desempeñará y evolucionará en el mercado, lo que facilitará la anticipación de utilidades y/o pérdidas que puedan surgir, identificando áreas de oportunidad, fallos y posibles mejoras. En la tabla \ref{proyeccion} se presenta una proyección promedio de ventas por mes y por año para los planes ofrecidos.

\vspace{2mm}
\begin{minipage}{0.9\textwidth}
\centering
\captionof{table}[{Proyección ventas}]{ Proyección ventas. }
\label{proyeccion}
\includegraphics[width=0.9\textwidth]{Images/proyeccionVentas2.png}
\fnote{Nota. \textup{Fuente : Autores}}
\end{minipage}

Se utiliza una proyección del IPC para Colombia en los próximos 5 años para determinar los flujos de efectivo que podría manejar la empresa con el plan de negocio. 
En la tabla \ref{ipc} se indexa la información mencionada.

\vspace{2mm}
\begin{minipage}{0.9\textwidth}
\centering
\captionof{table}[{IPC}]{ IPC. }
\label{ipc}
\includegraphics[width=0.9\textwidth]{Images/IPC.png}
\fnote{Nota. \textup{Fuente : Autores}}
\end{minipage}