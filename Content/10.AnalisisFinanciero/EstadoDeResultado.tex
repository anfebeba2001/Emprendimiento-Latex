\subsection{Estado de resultados}

El estado de resultados está proyectado a 5 años, lo que permite conocer en detalle los movimientos de dinero dentro de la empresa, considerando los costos necesarios para generar las utilidades netas del año fiscal.

En la tabla \ref{estado} se muestra la utilidad bruta, calculada al restar las ventas netas del total de costos de ventas. Además, se presenta la utilidad operacional, que se obtiene al restar los gastos administrativos de la utilidad bruta. Por último, se consideran los gastos financieros y pre-operativos, que determinan la utilidad antes de impuestos, teniendo en cuenta la tasa de impuestos sobre la renta correspondiente al año gravable 2024.

\vspace{2mm}
\begin{minipage}{0.9\textwidth}
\centering
\captionof{table}[{Estado de resultados }]{ Estado de resultados }
\label{estado}
\includegraphics[width=1.1\textwidth]{Images/estadoResultados.png}
\fnote{Nota. \textup{Fuente : Autores}}
\end{minipage}
\newpage