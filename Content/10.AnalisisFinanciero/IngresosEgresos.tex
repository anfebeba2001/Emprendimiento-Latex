\subsection{Ingresos y egresos}

\subsubsection{Ingresos}

El modelo de ingresos contempla los planes de inversión de personas naturales y de empresas. Las personas naturales y las empresas pueden usar el plan sin registrarse, pero sin muchos beneficios. Las personas naturales pueden ingresar a un plan premium donde reciben recomendaciones mucho más personalizadas. Las empresas pueden pagar por un plan mediano para acceder a estadísticas básicas de sus establecimientos o pagar extra para mostrar publicidad y obtener aún más información sobre su empresa a ojos del turista.

\vspace{2mm}
\begin{minipage}{0.9\textwidth}
\centering
\captionof{table}[{Ingresos}]{Ingresos}
\label{Ingresos}
\includegraphics[width=0.6\textwidth]{Content/Images/AF/ingresos.png}
\footnote{Nota. \textup{Fuente: Autores.}}
\end{minipage}

\subsubsection{Egresos}
Durante el primer año, el equipo estará conformado por las dos personas que lideran el proyecto, quienes recibirán su remuneración conforme a las obligaciones legales y aportes parafiscales establecidos en Colombia. También se incluyen los costos asociados a la infraestructura virtual necesaria para operar. La tabla \ref{egresos} muestra un desglose detallado de estos gastos.

\vspace{2mm}
\begin{minipage}{0.9\textwidth}
\centering
\captionof{table}[{Egresos}]{Egresos}
\label{egresos}
\includegraphics[width=0.6\textwidth]{Content/Images/AF/egresos.png}
\footnote{Nota. \textup{Fuente: Autores.}}
\end{minipage}
