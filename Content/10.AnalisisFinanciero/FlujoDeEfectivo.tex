\subsection{Flujo de efectivo}

El flujo de efectivo permite evaluar la capacidad de la empresa para generar fondos suficientes, de modo que pueda cumplir con sus compromisos, realizar inversiones y llevar a cabo sus procesos de expansión.

Existen tres tipos de flujos de caja que se deben considerar para obtener el flujo neto:

\begin{itemize}
    \item \textbf{Flujo de caja de operación: } Se obtiene a partir de la utilidad generada en el estado de resultados, incluyendo ajustes por depreciación, amortización, intereses de la deuda e impuestos sobre la renta.
    \item \textbf{Flujo de caja de inversión: } Incluye todas las inversiones realizadas para el funcionamiento de la empresa, abarcando tanto las inversiones en activos tangibles como intangibles.
    \item \textbf{Flujo de caja de financiamiento: } Se refiere al dinero que proviene de inversionistas y a los pagos asociados a ese financiamiento.
\end{itemize}

La suma total de estos tres tipos de flujos proporciona el resultado final de la cantidad de dinero que la empresa generó durante el año analizado.

\vspace{2mm}
\begin{minipage}{0.8\textwidth}
\centering
\captionof{table}[{Flujo de caja de efectivo }]{ Flujo de caja de efectivo}
\label{flujoOperacional}
\includegraphics[width=1.2\textwidth]{Images/flujoEfectivo.png}
\fnote{Nota. \textup{Fuente : Autores}}
\end{minipage}
