\subsection{Indicadores Financieros}

Los indicadores financieros se fundamentan en la comparación de dos elementos clave dentro de los documentos internos de una empresa: el balance general y el presupuesto. A través de estos indicadores y su interpretación, un negocio puede determinar qué dirección tomar basándose en el análisis de datos históricos.

Los indicadores financieros considerados en este análisis fueron los siguientes:


\begin{itemize}
    \item \textbf{Razón corriente: } Evalúa la capacidad de la empresa para cubrir sus deudas utilizando los activos disponibles.
    
    \item \textbf{Nivel de endeudamiento total: } Determina el grado y la forma en que los acreedores participan en la economía de la empresa.
    
    \item \textbf{Rentabilidad operacional: } : Muestra el porcentaje de ingresos que se convierte en beneficios después de pagar los costos operativos.
    
    \item \textbf{Rentabilidad neta: } Indica el nivel de ganancias obtenidas en relación con las ventas netas, una vez que se han pagado los costos fijos y variables.
    
    \item \textbf{Rentabilidad de patrimonio: } Refleja el nivel de ganancias que la empresa genera a partir de la inversión realizada por los accionistas.
    
    \item \textbf{Rentabilidad del activo:} Mide la rentabilidad de los activos adquiridos por la empresa y cómo estos generan beneficios a lo largo del tiempo.
\end{itemize}

En la tabla \ref{indicadores} se proyectan los indicadores a 5 años.

\vspace{2mm}
\begin{minipage}{0.9\textwidth}
\centering
\captionof{table}[{Indicadores financieros}]{ Indicadores financieros}
\label{indicadores}
\includegraphics[width=0.9\textwidth]{Images/indicadoresFinancieros.png}
\fnote{Nota. \textup{Fuente : Autores}}
\end{minipage}

Además, se emplearon los indicadores financieros VAN, TIR y la relación costo-beneficio con el objetivo de mostrar la viabilidad de la inversión para los accionistas. Los resultados obtenidos del flujo de caja se consideraron para los datos presentados en la tabla \ref{flujoLibre}.

\vspace{2mm}
\begin{minipage}{0.9\textwidth}
\centering
\captionof{table}[{Flujo de caja libre}]{ Flujo de caja libre}
\label{flujoLibre}
\includegraphics[width=\textwidth]{Images/flujoCajaLibre.png}
\fnote{Nota. \textup{Fuente : Autores}}
\end{minipage}

Considerando la tasa interna de oportunidad, que representa la tasa de rendimiento mínima que los accionistas esperan recibir, se asigna de manera subjetiva un valor del 10\% como compensación por la inversión realizada.

La tabla \ref{vanTIR} presenta los resultados que demuestran que la inversión se recupera en un periodo de 5 años, excediendo los \$343.585.469 y multiplicando el capital invertido por más de 5 veces, con una tasa interna de retorno del 107\%.

\vspace{2mm}
\begin{minipage}{0.9\textwidth}
\centering
\captionof{table}[{VAN, TIR y RBC}]{ VAN, TIR y RBC}
\label{vanTIR}
\includegraphics[width=\textwidth]{Images/vanTIR.png}
\fnote{Nota. \textup{Fuente : Autores}}
\end{minipage}

El análisis de los resultados financieros confirma la viabilidad del proyecto. Su éxito dependerá del interés que despierte entre los inversionistas, lo cual permitirá a la empresa ampliar su capacidad en un mercado altamente competitivo y con una demanda significativa en este sector. Tanto el Valor Actual Neto (VAN) como la Tasa Interna de Retorno (TIR) son positivos, lo que, sumado a una proyección de incremento en la liquidez según la tabla de ventas, sugiere una mejora considerable en la capacidad financiera de la empresa. Además, se anticipa una reducción en el nivel de endeudamiento y un crecimiento exponencial de la rentabilidad en los próximos años.