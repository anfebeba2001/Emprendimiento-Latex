\subsection{Descripción del Problema}
Varias compañías del sector de la restauración–Partoo, Superpopi, Healthy Poke, Talent Class y Localboss– se han posicionado públicamente en contra del filtrado de reseñas. Lo consideran como una manipulación que ofrece una visión distorsionada de la realidad a los consumidores. El año pasado, Google eliminó 170 millones de reseñas y 12 millones de perfiles de empresa que no cumplían con su política de contenido, lo que supone un 43\% más que en 2022, según ha contado recientemente la compañía en su blog.\cite{hotelesReseñasNeg}

Esta declaración deja al descubierto como muchas empresas a sabiendas de que es mas fácil modificar deshonestamente sus reseñas haciéndose pasar por clientes conocedores en el tema del turismo que arreglar sus establecimientos y mejorar sus servicios para que cumplan con las expectativas que le plantean en su publicidad.  

Por el otro lado, otra problemática que también se planea mitigar a nivel local con esta oportunidad de desarrollo es la desmeritar reseñas y opiniones que estén fuera de lugar o estén sesgadas negativamente, entiéndase como una “reseña sesgada negativamente” aquella que no abarca una calificación correspondiente a los productos, servicios o experiencia general del cliente ofrecido directamente por una compañía, por ejemplo, no es justo para el emprendedor local que su negocio sea negativamente calificado porque la tienda de al lado hace ruido o el día que estaba consumando su estadía hubo lluvias y no fue de su agrado, este tipo de reseñas perjudica de sobremanera a un establecimiento y esto se evidencia en trabajos de investigación como el realizado por la revista digital Partoo donde se entrevistó a consumidores sobre su frecuencia a revisar reseñas y se llega a la conclusión de que “El 75\% de los encuestados admite que nunca elige un establecimiento con una valoración inferior a 3.5/5” 