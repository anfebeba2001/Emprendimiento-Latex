\section{Resumen}
\noindent
El propósito del siguiente documento es presentar y justificar el plan de negocio para el desarrollo de una plataforma de reseñas y recomendaciones personalizadas enfocadas a los nómadas digitales en la ciudad de Bogotá. Haciendo énfasis en la seguridad y la comodidad de los usuarios, proporcionando recomendaciones personalizadas basadas en gustos personales, reseñas de usuarios y estudios de seguridad en la zona.

Se implementó la metodología Canvas para estructurar el plan de negocio, definiendo así la propuesta de valor, los segmentos de clientes, los canales de distribución, las fuentes de ingresos y los recursos clave necesarios para el desarrollo de la plataforma. Esta metodología permitió identificar de manera clara los elementos fundamentales para la viabilidad y sostenibilidad del proyecto.
Esto apoyado en el análisis de la matriz DOFA y PEYEA, permitiendo evaluar los ámbitos internos y externos, planteando estrategias que permitan acelerar y garantizar el progreso del proyecto. Además, se realizó el análisis financiero que justifica la viabilidad del proyecto.

Adicionalmente, se identificaron oportunidades de crecimiento y diferenciación frente a la competencia mediante la integración de tecnologías innovadoras y la personalización de los servicios ofrecidos. Se espera que la plataforma sea un referente en el mercado turístico y de tecnología, por su seguridad, fiabilidad y enfoque en la excelente experiencia de usuario.