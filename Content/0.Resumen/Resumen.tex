 %esto es lo nuevo que agregue
\phantomsection\section*{Resumen}
\addcontentsline{toc}{section}{Resumen}
\noindent 

% Motivo del documento, enfasis en el plan de negocios, descripcion de los procesos y resultados esperados, metodologias y trabajo realizado (100 - 150 palabras)

\textcolor{red}{El propósito de este documento es presentar el plan de negocios para la creación de una plataforma de entrenamiento personalizado en línea en la ciudad de Bogotá. Este proyecto se enfoca en  mejorar la calidad de vida de los usuarios, proporcionando programas de entrenamiento que se adapten a las necesidades individuales de cada persona. La plataforma está diseñada para individuos que deseen mejorar su bienestar físico y mental. }

\textcolor{red}{Para estructurar el plan de manera adecuada, se utilizó la metodología canvas, lo cual permitió definir claramente la propuesta de valor y los requisitos necesarios para su implementación. Además,  se llevó a cabo un análisis PEYEA/DOFA, evaluando las fortalezas y el posicionamiento de la propuesta en comparación con otros competidores, demostrando una alta viabilidad desde los puntos de vista financiero y de mercado.}

\textcolor{red}{Como resultado, se espera que la plataforma mejore significativamente la eficiencia de los programas de entrenamiento, ofreciendo un manejo óptimo de la información y procesos. A largo plazo, se anticipa que esto aumentará la satisfacción y el éxito de los usuarios. }