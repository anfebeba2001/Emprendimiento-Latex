\section{Introducción}
Bogotá, por su posición geográfica y las 122 mil hectáreas de suelo rural, goza de una invaluable riqueza natural debido a la confluencia de ecosistemas de todo tipo, pues es posible encontrar cerros, llanuras, bosques, humedales, producciones agropecuarias, centros poblados, lagunas, parques y senderos, pudiendo combinar un turismo de naturaleza con montaña, humedales y corredores ecológicos~\cite{EstudioTurismoenterritoriosruralesdeBogotá}.

Para el desarrollo de las actividades de turismo en territorio bogotano es importante comprender a profundidad el panorama de las personas que promueven o realizan esta actividad, dado que es un campo que presenta tanta variedad y riqueza, que tiene ofertas y modalidades que pasan desapercibidas para potenciales consumidores.

Bajo este marco, la nueva ruralidad permite desarrollar una amplia variedad de actividades, empoderando a las comunidades rurales a través de mayores oportunidades de empleo, destacando a los jóvenes y mujeres, favoreciendo la preservación de los hábitats y la conservación del patrimonio natural y cultural de las regiones~\cite{EstudioTurismoenterritoriosruralesdeBogotá}.

Comprendiendo el marco rural de la ciudad, se debe hacer énfasis también en el mercado hotelero y gastronómico, ya que actualmente existen aplicaciones y sitios web con opciones y lugares prometedores para explotar el turismo bogotano. Pese a estas herramientas tecnológicas, la comodidad del usuario final como consumidor o prestador de servicio que quiere darse a conocer no es muy cercana u honesta, ya que muchas plataformas tergiversan las reseñas y manipulan fácilmente a los usuarios finales, influyendo negativamente en su experiencia final.

Es imperativo crear una plataforma que permita dar a conocer el mercado turístico tanto rural como urbano de la ciudad y reseñarlas de forma honesta y transparente.
