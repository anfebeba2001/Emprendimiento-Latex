\subsection{Analisis de competencia}

\subsubsection*{Airbnb}
Airbnb comenzó su camino en 2008, cuando dos diseñadores que tenían espacio libre en casa alojaron a tres viajeros que buscaban un lugar en donde quedarse. En la actualidad, millones de anfitriones y huéspedes han creado cuentas gratuitas en Airbnb para disfrutar su visión compartida del mundo.
Desde acogedoras casa de campo hasta Penthouses elegantes, los anfitriones estarán encantados de compartir sus espacios. Ya sea un viaje de trabajo, una escapada de fin de semana, unas vacaciones en familia o una estadía más larga, hay millones de lugares increíbles para visitar.
También hay guías locales que llevan grupos pequeños a los territorios más inexplorados en expediciones totalmente inmersivas de Aventuras en Airbnb.
fuente:https://www.airbnb.com.co/help/article/2503

\subsubsection*{Rappi travel}
Rappi Travel facilita la intermediación entre los Usuarios de los servicios de hotelería, vuelos, y cualquier otro tipo de servicio asociado al turismo (colectivamente “Servicios Turísticos”) y prestadores y proveedores de Servicios Turísticos (“Proveedor(es)”). Por intermedio de Rappi Travel, los Proveedores podrán ofrecer y publicitar los Servicios Turísticos que brindan y, por su parte, los Usuarios podrán reservar dichos Servicios Turísticos en tiempo real y contratarlos por separado o combinados, armando y gestionando su propio viaje, de conformidad con sus intereses particulares.
fuente:https://travel.rappi.com.mx/terms

\subsubsection*{Nomad List}
Nomad List es una base de datos de ciudades para teletrabajadores y nómadas digitales. Permite filtrar por una enorme base de datos para encontrar los mejores precios, el clima, la velocidad de internet o incluso la calidad del aire en ciudades de todo el mundo.
Fue creada en 2014 por Pieter Levels y, desde entonces, ha crecido hasta alcanzar los 700.000 ingresos anuales recurrentes (ARR).

Pero empezó siendo algo mucho más modesto: una simple hoja de cálculo que se hizo pública y se compartió en Twitter.

Se volvió viral y Pieter pudo crear una primera versión de Nomad List con todos los datos que obtuvo mediante crowdsourcing y lanzar un MVP completo en menos de un mes.
fuente:https://www.softwaregrowth.io/blog/how-pieter-levels-grew-nomad-list

\section{Antecedentes}

\subsection{Antecedentes Nacionales (Colombia): }

\begin{itemize}
\item Rappi Travel: La plataforma de entrega a domicilio Rappi ha lanzado un servicio de viaje dentro de su aplicación, permitiendo a los usuarios acceder a recomendaciones de destinos, alojamiento y transporte. Aunque no está específicamente dirigida a nómadas digitales, este servicio analiza preferencias y comportamiento de los usuarios para sugerir opciones personalizadas, lo que puede relacionarse con la idea de crear recomendaciones turísticas basadas en factores culturales y geográficos. 

\item TripTrip: Una aplicación colombiana que facilita la planificación de viajes personalizados en el país. Aunque está más centrada en viajeros comunes, utiliza la integración de mapas y puntos de interés, lo que podría servir de referencia para el desarrollo de una plataforma orientada a nómadas digitales en Bogotá. 

\end{itemize}


\subsection{Antecedentes Internacionales:  }

\begin{itemize}
\item Nomad List: Una de las plataformas más populares entre nómadas digitales a nivel global, que ofrece una clasificación de ciudades según factores como costo de vida, seguridad, calidad de internet, clima y comunidad de expatriados. Aunque es una plataforma internacional, se centra en ofrecer recomendaciones confiables para nómadas digitales, lo cual tiene puntos en común con el proyecto planteado para Bogotá. 

\item Airbnb Experiences: Aunque Airbnb es conocida por su plataforma de alojamiento, la iniciativa "Airbnb Experiences" permite a los viajeros, incluidos nómadas digitales, descubrir actividades organizadas por locales en diferentes ciudades del mundo. Esta plataforma también considera factores culturales y geográficos en sus recomendaciones. 

\item Workfrom: Esta plataforma internacional está enfocada en ayudar a los nómadas digitales a encontrar espacios de trabajo cómodos y seguros en distintas ciudades. Utiliza la geolocalización y la reseña de otros usuarios para proporcionar sugerencias basadas en la conectividad, el ambiente de trabajo y la seguridad, lo cual es relevante para el enfoque de la aplicación en Bogotá. 

Estos antecedentes muestran la creciente importancia de las aplicaciones que facilitan la vida de los nómadas digitales al proporcionar recomendaciones basadas en reseñas y factores locales, como la seguridad y la comodidad. 
\end{itemize}