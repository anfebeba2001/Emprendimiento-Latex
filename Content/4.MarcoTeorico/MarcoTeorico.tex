\section{Marco Teórico}

\subsection{Nómada digital}

 Un nómada digital es una persona que utiliza la tecnología para trabajar de forma remota mientras viaja y cambia de ubicación regularmente. Este estilo de vida es posible gracias al acceso a internet, computadoras portátiles y plataformas digitales, lo que les permite a los nómadas digitales trabajar desde cualquier parte del mundo. Los trabajos más comunes entre los nómadas digitales suelen estar en la economía del conocimiento, como diseño, marketing, programación, escritura y consultoría.\cite{NomDig}



\subsection{Metodología ágil}

La metodología ágil es un enfoque para la gestión de proyectos, principalmente en el desarrollo de software, que se caracteriza por su flexibilidad, colaboración continua y respuesta rápida al cambio. Su objetivo es entregar productos funcionales en iteraciones cortas, conocidas como sprints, en lugar de completar todo el proyecto de una sola vez. 

Dentro del marco ágil, hay varias metodologías, como Scrum, Kanban y Extreme Programming (XP). Estas metodologías comparten principios fundamentales, como la entrega de valor al cliente de manera frecuente, la colaboración constante entre los equipos de trabajo y la adaptación continua a los cambios que puedan surgir durante el proceso de desarrollo. \cite{MetAgil}

\subsection{Metodologia Scrum }

Scrum es un marco de trabajo ágil para gestionar proyectos, particularmente en el desarrollo de software, aunque su aplicación se ha expandido a otras áreas. Se basa en principios de colaboración, iteración y entrega incremental de valor. El proceso de Scrum divide el trabajo en ciclos cortos y repetitivos llamados sprints, típicamente de 2 a 4 semanas de duración, durante los cuales los equipos trabajan en alcanzar metas específicas. Al final de cada sprint, se revisa el progreso y se ajustan los objetivos si es necesario.\cite{Scrum}

\subsection{Página web }

Una página web es un documento digital accesible a través de internet, compuesto principalmente de texto, imágenes, videos y enlaces, creado usando lenguajes de marcado como HTML, CSS y JavaScript. Cada página web tiene una dirección única (URL) que permite a los usuarios acceder a ella a través de un navegador web. Las páginas web forman parte de un sitio web, que puede contener varias páginas interrelacionadas. Estas páginas pueden ser estáticas (mostrando siempre el mismo contenido) o dinámicas (modificando su contenido según las interacciones del usuario o de un servidor). \cite{web}

\subsection{Analisis de competencia}

\subsubsection*{Airbnb}
Airbnb comenzó su camino en 2008, cuando dos diseñadores que tenían espacio libre en casa alojaron a tres viajeros que buscaban un lugar en donde quedarse. En la actualidad, millones de anfitriones y huéspedes han creado cuentas gratuitas en Airbnb para disfrutar su visión compartida del mundo.
Desde acogedoras casa de campo hasta Penthouses elegantes, los anfitriones estarán encantados de compartir sus espacios. Ya sea un viaje de trabajo, una escapada de fin de semana, unas vacaciones en familia o una estadía más larga, hay millones de lugares increíbles para visitar.
También hay guías locales que llevan grupos pequeños a los territorios más inexplorados en expediciones totalmente inmersivas de Aventuras en Airbnb.
fuente:https://www.airbnb.com.co/help/article/2503

\subsubsection*{Rappi travel}
Rappi Travel facilita la intermediación entre los Usuarios de los servicios de hotelería, vuelos, y cualquier otro tipo de servicio asociado al turismo (colectivamente “Servicios Turísticos”) y prestadores y proveedores de Servicios Turísticos (“Proveedor(es)”). Por intermedio de Rappi Travel, los Proveedores podrán ofrecer y publicitar los Servicios Turísticos que brindan y, por su parte, los Usuarios podrán reservar dichos Servicios Turísticos en tiempo real y contratarlos por separado o combinados, armando y gestionando su propio viaje, de conformidad con sus intereses particulares.
fuente:https://travel.rappi.com.mx/terms

\subsubsection*{Nomad List}
Nomad List es una base de datos de ciudades para teletrabajadores y nómadas digitales. Permite filtrar por una enorme base de datos para encontrar los mejores precios, el clima, la velocidad de internet o incluso la calidad del aire en ciudades de todo el mundo.
Fue creada en 2014 por Pieter Levels y, desde entonces, ha crecido hasta alcanzar los 700.000 ingresos anuales recurrentes (ARR).

Pero empezó siendo algo mucho más modesto: una simple hoja de cálculo que se hizo pública y se compartió en Twitter.

Se volvió viral y Pieter pudo crear una primera versión de Nomad List con todos los datos que obtuvo mediante crowdsourcing y lanzar un MVP completo en menos de un mes.
fuente:https://www.softwaregrowth.io/blog/how-pieter-levels-grew-nomad-list


