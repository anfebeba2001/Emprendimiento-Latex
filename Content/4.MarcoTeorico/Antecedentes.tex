\section{Antecedentes}

\subsection{Antecedentes Nacionales (Colombia): }

\begin{itemize}
\item Rappi Travel: La plataforma de entrega a domicilio Rappi ha lanzado un servicio de viaje dentro de su aplicación, permitiendo a los usuarios acceder a recomendaciones de destinos, alojamiento y transporte. Aunque no está específicamente dirigida a nómadas digitales, este servicio analiza preferencias y comportamiento de los usuarios para sugerir opciones personalizadas, lo que puede relacionarse con la idea de crear recomendaciones turísticas basadas en factores culturales y geográficos. 

\item TripTrip: Una aplicación colombiana que facilita la planificación de viajes personalizados en el país. Aunque está más centrada en viajeros comunes, utiliza la integración de mapas y puntos de interés, lo que podría servir de referencia para el desarrollo de una plataforma orientada a nómadas digitales en Bogotá. 

\end{itemize}


\subsection{Antecedentes Internacionales:  }

\begin{itemize}
\item Nomad List: Una de las plataformas más populares entre nómadas digitales a nivel global, que ofrece una clasificación de ciudades según factores como costo de vida, seguridad, calidad de internet, clima y comunidad de expatriados. Aunque es una plataforma internacional, se centra en ofrecer recomendaciones confiables para nómadas digitales, lo cual tiene puntos en común con el proyecto planteado para Bogotá. 

\item Airbnb Experiences: Aunque Airbnb es conocida por su plataforma de alojamiento, la iniciativa "Airbnb Experiences" permite a los viajeros, incluidos nómadas digitales, descubrir actividades organizadas por locales en diferentes ciudades del mundo. Esta plataforma también considera factores culturales y geográficos en sus recomendaciones. 

\item Workfrom: Esta plataforma internacional está enfocada en ayudar a los nómadas digitales a encontrar espacios de trabajo cómodos y seguros en distintas ciudades. Utiliza la geolocalización y la reseña de otros usuarios para proporcionar sugerencias basadas en la conectividad, el ambiente de trabajo y la seguridad, lo cual es relevante para el enfoque de la aplicación en Bogotá. 

Estos antecedentes muestran la creciente importancia de las aplicaciones que facilitan la vida de los nómadas digitales al proporcionar recomendaciones basadas en reseñas y factores locales, como la seguridad y la comodidad. 
\end{itemize}