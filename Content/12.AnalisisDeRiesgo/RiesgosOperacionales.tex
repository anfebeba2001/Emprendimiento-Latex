
\color{red}
\subsection{Riesgos operacionales}

Según Ian Storkey, los riesgos operacionales abarcan una amplia variedad de amenazas, que incluyen la pérdida de personal clave, fallas en la liquidación de pagos e incumplimientos, así como el robo, fallos en los sistemas y daños a las instalaciones. En este contexto, la gestión del riesgo operacional tiene como objetivo asegurar la integridad y la calidad de las operaciones del Ministerio de Hacienda y las actividades de tesorería, utilizando diversas herramientas como auditorías, políticas de contratación, sistemas de control y planes de continuidad operativa. \cite{Storkey}.

\textbf{Riesgo:} Fallos en los sistemas de cómputo.

\textbf{Contra medida:} Se implementará una solución de computación en la nube, que proporciona escalabilidad, alta disponibilidad y seguridad, ayudando a mitigar el riesgo de interrupciones en el servicio, pérdida de datos y otros problemas. Además, se establecerán acuerdos de nivel de servicio (SLA) con el proveedor de la nube para asegurar el rendimiento y la confiabilidad de la plataforma.

\textbf{Riesgo:} Daño o pérdida de información.

\textbf{Contra medida:} Se establecerán políticas y procedimientos de seguridad de la información que incluirán medidas de seguridad física, como controles de acceso a los centros de datos, y medidas de seguridad lógica, como firewalls, antivirus y software de detección de intrusiones. También se implementarán medidas de seguridad administrativa, que contemplarán capacitación en seguridad para el personal, planes de respaldo y recuperación de desastres, así como controles de acceso a la información.

\textbf{Riesgo:} Infraestructura insuficiente. 

\textbf{Contra medida:} Se llevará a cabo una planificación y estimación de recursos adecuada antes de iniciar cualquier nuevo proyecto. Este proceso incluirá una evaluación de las necesidades de hardware, software y personal para respaldar el crecimiento y la demanda de la plataforma. Se buscarán soluciones de infraestructura escalables y flexibles que se adapten a las necesidades cambiantes del proyecto.

\textbf{Riesgo:} Gestión inadecuada de proyectos. 

\textbf{Contra medida:} Se implementarán metodologías y herramientas de gestión de proyectos comprobadas. Estas ayudarán a planificar de manera cuidadosa los proyectos, estimar los recursos necesarios, asignar responsabilidades de forma clara y monitorear su progreso. Además, se establecerán mecanismos de comunicación y colaboración efectivos entre los equipos para asegurar que todos estén informados y comprometidos con el proyecto.

\textbf{Riesgo:} Falta de capacitación y desarrollo del personal.

\textbf{Contra medida:} Se diseñarán programas de capacitación y desarrollo del personal que aborden las necesidades específicas de habilidades y conocimientos requeridos para las operaciones comerciales. Además, se llevarán a cabo evaluaciones de desempeño periódicas, proporcionando retroalimentación y oportunidades de crecimiento profesional para fomentar la retención del talento y mejorar la eficiencia operativa.

\textbf{Riesgo:} Falta de recursos humanos o financieros.

\textbf{Contra medida:} Se establecerá un plan de gestión de recursos humanos y financieros que incluirá la identificación de las necesidades de recursos para el proyecto, la elaboración de un presupuesto y la búsqueda de fuentes de financiamiento. Se implementarán procesos de contratación y capacitación para asegurar que el personal del proyecto esté calificado y posea las habilidades necesarias para realizar el trabajo. Además, se supervisará el uso de los recursos y se realizarán ajustes al plan según sea necesario.
