\subsection{Factores limitantes y obstáculos}
Al desarrollar un proyecto empresarial, es crucial reconocer y abordar los factores limitantes y obstáculos que pueden surgir durante su implementación. Estos elementos pueden influir significativamente en el éxito del proyecto y deben ser gestionados de manera proactiva para minimizar su impacto negativo.
Por lo tanto, es importante identificar y analizar estos factores desde el inicio del proyecto, para poder establecer estrategias que permitan superarlos o mitigarlos. Algunos de los factores limitantes comunes incluyen la falta de recursos financieros, la resistencia al cambio por parte del equipo, la falta de habilidades técnicas específicas y las limitaciones impuestas por el entorno regulatorio o competitivo. Además, es fundamental considerar los obstáculos internos, como la falta de comunicación efectiva entre los miembros del equipo o la ausencia de una visión clara y compartida del proyecto.
