
\color{red}
\subsection{Riesgos legales}

El riesgo legal se refiere a la posibilidad de incumplir leyes, normativas y regulaciones emitidas por los gobiernos de cada país o por otras entidades. Este riesgo puede surgir tanto por desconocimiento de una ley o norma específica como por su omisión deliberada. Asimismo, el riesgo legal también abarca el incumplimiento de contratos y acuerdos comerciales con terceros \cite{pira}.

\textbf{Riesgo:} Protección de la empresa ineficiente.

\textbf{Mitigación:} Se establecerán políticas internas para garantizar un manejo correcto de la propiedad intelectual, marcas registradas, bases de datos y otros activos de la empresa. Además, se contará con un equipo legal especializado para asegurar la adecuada protección de los derechos y activos de la organización.

\textbf{Riesgo:} Reclamos y acciones legales.

\textbf{Mitigación:} Se contratarán servicios de asesoría legal especializada para gestionar cualquier reclamación o acción legal que pueda surgir. Estos profesionales representarán a la empresa en procesos legales, investigaciones y procedimientos administrativos, contribuyendo a minimizar el riesgo de multas y sanciones.

\textbf{Riesgo:} Uso de software sin licencia.


\textbf{Mitigación:} Se establecerán políticas claras de licenciamiento de software que regulen el uso adecuado de programas, tanto libres como de pago. Un equipo especializado se encargará de supervisar el cumplimiento de estas políticas y garantizar que todos los programas utilizados cuenten con la licencia correspondiente.

\textbf{Riesgo:} Incumplimiento de contratos.


\textbf{Mitigación:} Se buscará la asesoría de profesionales legales para redactar y ejecutar contratos, asegurando que estos reflejen de manera clara y precisa los términos y condiciones de los acuerdos comerciales. Además, se establecerán procesos internos para garantizar que la empresa cumpla con sus compromisos contractuales.

\textbf{Riesgo:} Disputas contractuales con proveedores o clientes.


\textbf{Mitigación:} Se establecerán cláusulas claras y detalladas en los contratos con proveedores y clientes, definiendo los derechos y responsabilidades de ambas partes, así como los procedimientos para la resolución de disputas. Además, se designará un punto de contacto específico para gestionar las comunicaciones y resolver cualquier inconveniente de manera oportuna y efectiva.

\textbf{Riesgo:} Publicidad Engañosa.


\textbf{Mitigación:} Para prevenir la publicidad engañosa o falsa, se garantizará que toda la comunicación publicitaria sea veraz y precisa. No se realizarán promesas que no puedan cumplirse ni se empleará información falsa o engañosa para atraer a los usuarios. Además, se implementará un sistema de revisión y aprobación de la publicidad para asegurar que cumpla con los estándares establecidos.