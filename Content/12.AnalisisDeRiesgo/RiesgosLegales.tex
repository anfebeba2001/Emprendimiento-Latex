\subsection{Riesgos legales}
Los riesos legales son aquellos que pueden surgir debido a la falta de cumplimiento de leyes, normativas y regulaciones establecidas por los gobiernos o entidades competentes. Cualquier tipo de incumplimiento, ya sea por desconocimiento o por omisión deliberada, puede acarrear consecuencias legales significativas para la organización.

\textbf{Riesgo:}Incumplimiento de normativas de protección de datos 

\textbf{Mitigación} Implementar políticas de privacidad y protección de datos que cumplan con las normativas vigentes, como el GDPR o la Ley de Protección de Datos Personales. Esto incluye la capacitación del personal en el manejo adecuado de datos sensibles y la implementación de medidas de seguridad para proteger la información.

\textbf{Riesgo:}Infracciones laborales en contratación y despido

\textbf{Mitigación:} Establecer procedimientos claros y transparentes para la contratación y despido de empleados, asegurando el cumplimiento de las leyes laborales locales. Esto incluye la revisión de contratos laborales, la capacitación del personal de recursos humanos y la implementación de políticas de igualdad y no discriminación.

\textbf{Riesgo:}Violación de normas antitrus/competencia desleal
\textbf{Mitigación:} Implementar políticas de cumplimiento normativo que incluyan la capacitación del personal sobre las leyes antimonopolio y de competencia desleal. Además, se deben establecer mecanismos de monitoreo y auditoría para detectar y prevenir prácticas comerciales desleales.

\textbf{Riesgo:}Incumplimiento de regulaciones ambientales

\textbf{Mitigación:} Desarrollar un plan de gestión ambiental que cumpla con las regulaciones locales e internacionales. Esto incluye la identificación de impactos ambientales, la implementación de medidas de mitigación y la capacitación del personal en prácticas sostenibles.

\textbf{Riesgo:}Fiscalización tributaria agresiva

\textbf{Mitigación:} Mantener una contabilidad transparente y cumplir con todas las obligaciones fiscales. Esto incluye la contratación de asesores fiscales para garantizar el cumplimiento de las leyes tributarias y la preparación adecuada de declaraciones fiscales.

\textbf{Riesgo:}Vulneraciones de seguridad cibernética

\textbf{Mitigación:} Implementar medidas de seguridad cibernética robustas, como firewalls, sistemas de detección de intrusos y capacitación del personal en prácticas seguras de manejo de información. Además, se deben realizar auditorías regulares de seguridad para identificar y corregir vulnerabilidades.
%----------------------------------------------------