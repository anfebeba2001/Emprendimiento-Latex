\section{Análisis de riesgo}
De acuerdo con el Artículo 1 de la Ley 1523 de 2012, la gestión del riesgo de desastres es un proceso social orientado a la formulación, ejecución, seguimiento y evaluación de políticas, estrategias, planes, programas, regulaciones, instrumentos, medidas y acciones permanentes para el conocimiento y la reducción del riesgo y para el manejo de desastres, con el propósito explícito de contribuir a la seguridad, el bienestar, la calidad de vida de las personas y al desarrollo sostenible. 

El análisis de riesgo es un proceso que permite identificar, evaluar y priorizar los riesgos asociados a una organización o proyecto. Este proceso es esencial para la toma de decisiones informadas y para la implementación de medidas que minimicen los impactos negativos de los riesgos identificados.

\color{red}

\subsection{Factores limitantes y obstáculos}

Los proyectos empresariales suelen enfrentar una serie de desafíos y obstáculos que forman parte integral de su evolución. Estos retos pueden surgir de la interacción entre el equipo humano, los procesos operativos y las soluciones tecnológicas implementadas. A lo largo del tiempo, esta realidad ha impactado a empresas de todo tipo, manifestándose de diversas maneras y teniendo causas múltiples. La frecuente aparición de pérdidas significativas en el ámbito empresarial indica una falta de comprensión sobre estos factores clave, así como la ausencia de métodos efectivos para su control y gestión. Por ello, es fundamental identificar y gestionar eficientemente estos elementos limitantes para mitigar su impacto negativo y asegurar el crecimiento y éxito de los proyectos empresariales.\cite{AdministrarProyectos}.


\color{red}
\subsection{Factores claves de éxito}

Los Factores Críticos de Éxito (FCE) son aquellas cualidades, recursos o capacidades altamente valorados por un segmento específico de clientes, y que una organización debe dominar para sobresalir frente a la competencia. Estos FCE constituyen los cimientos esenciales que soportan el éxito de una empresa en el mercado \cite{Asana}. Algunos de estos factores incluyen:

\begin{itemize}
    \item Estrategia de precios
    \item Estrategias de publicidad y marketing.
    \item Calidad del producto o servicio ofrecido. 
    \item Fidelización y satisfacción del cliente.
    \item Conocimiento profundo del mercado y sus necesidades.
    \item Planificación estratégica sólida. 
    \item Innovación continua en productos, servicios o procesos. 
    \item Capacidad de escalabilidad del producto o servicio para adaptarse al crecimiento y demanda del mercado. 
\end{itemize}


\color{red}
\subsection{Riesgos legales}

El riesgo legal se refiere a la posibilidad de incumplir leyes, normativas y regulaciones emitidas por los gobiernos de cada país o por otras entidades. Este riesgo puede surgir tanto por desconocimiento de una ley o norma específica como por su omisión deliberada. Asimismo, el riesgo legal también abarca el incumplimiento de contratos y acuerdos comerciales con terceros \cite{pira}.

\textbf{Riesgo:} Protección de la empresa ineficiente.

\textbf{Mitigación:} Se establecerán políticas internas para garantizar un manejo correcto de la propiedad intelectual, marcas registradas, bases de datos y otros activos de la empresa. Además, se contará con un equipo legal especializado para asegurar la adecuada protección de los derechos y activos de la organización.

\textbf{Riesgo:} Reclamos y acciones legales.

\textbf{Mitigación:} Se contratarán servicios de asesoría legal especializada para gestionar cualquier reclamación o acción legal que pueda surgir. Estos profesionales representarán a la empresa en procesos legales, investigaciones y procedimientos administrativos, contribuyendo a minimizar el riesgo de multas y sanciones.

\textbf{Riesgo:} Uso de software sin licencia.


\textbf{Mitigación:} Se establecerán políticas claras de licenciamiento de software que regulen el uso adecuado de programas, tanto libres como de pago. Un equipo especializado se encargará de supervisar el cumplimiento de estas políticas y garantizar que todos los programas utilizados cuenten con la licencia correspondiente.

\textbf{Riesgo:} Incumplimiento de contratos.


\textbf{Mitigación:} Se buscará la asesoría de profesionales legales para redactar y ejecutar contratos, asegurando que estos reflejen de manera clara y precisa los términos y condiciones de los acuerdos comerciales. Además, se establecerán procesos internos para garantizar que la empresa cumpla con sus compromisos contractuales.

\textbf{Riesgo:} Disputas contractuales con proveedores o clientes.


\textbf{Mitigación:} Se establecerán cláusulas claras y detalladas en los contratos con proveedores y clientes, definiendo los derechos y responsabilidades de ambas partes, así como los procedimientos para la resolución de disputas. Además, se designará un punto de contacto específico para gestionar las comunicaciones y resolver cualquier inconveniente de manera oportuna y efectiva.

\textbf{Riesgo:} Publicidad Engañosa.


\textbf{Mitigación:} Para prevenir la publicidad engañosa o falsa, se garantizará que toda la comunicación publicitaria sea veraz y precisa. No se realizarán promesas que no puedan cumplirse ni se empleará información falsa o engañosa para atraer a los usuarios. Además, se implementará un sistema de revisión y aprobación de la publicidad para asegurar que cumpla con los estándares establecidos.

\section{Riesgos operacionales}
El Comité de Basilea define al riesgo operacional como al riesgo de pérdidas resultantes de la falta de adecuación o fallas en los procesos internos, de la actuación del personal o de los sistemas o bien aquellas que sean producto de eventos externos.

\textbf{Riesgo:}Interrupciones en APIs de proveedores
\textbf{Contra medida:} Se implementará un sistema de caché inteligente para respuestas de APIs críticas y se establecerán conexiones redundantes con múltiples proveedores en cada categoría. Además, se desarrollará un módulo de "modo offline" que permita operaciones básicas cuando los servicios externos no estén disponibles, garantizando continuidad en la experiencia del usuario.

\textbf{Riesgo:} Inexactitud en información dinámica
\textbf{Contra medida:}Se implementará un sistema de verificación automatizada que cruce datos de múltiples fuentes en tiempo real y alertas de discrepancia. Además, se establecerá un protocolo de actualización horaria con proveedores y un equipo dedicado para corrección manual inmediata de datos críticos reportados por usuarios.

\textbf{Riesgo:}Sobrecarga en la plataforma
\textbf{Contra medida:} Se implementará un sistema de balanceo de carga avanzado que distribuya el tráfico de manera equitativa entre múltiples servidores. Además, se establecerán límites de uso por usuario y se implementarán mecanismos de escalabilidad automática para aumentar la capacidad durante picos de tráfico.

\textbf{Riesgo:}Vulneraciones en pagos en línea y datos sensibles de usuarios
\textbf{Contra medida:} Se integrarán soluciones de tokenización PCI-DSS para procesamiento de pagos y se implementará autenticación multifactor obligatoria para todas las cuentas con acceso a datos sensibles. Además, se realizarán auditorías de seguridad trimestrales por terceros especializados y cifrado de extremo a extremo para toda información financiera.
\textbf{Riesgo:}Contenido fraudulento o engañoso en reseñas y perfiles
\textbf{Contra medida:} Se implementará un sistema de verificación de identidad por dos pasos para creadores de contenido y algoritmos de detección de patrones fraudulentos. Además, se establecerá un equipo de moderación especializado y un sistema de reputación basado en interacciones validadas por la comunidad.

\textbf{Riesgo:}Fallos de integración de sistemas de partners
\textbf{Contra medida:} Se establecerán acuerdos de nivel de servicio (SLA) claros con todos los partners y se implementará un sistema de monitoreo continuo de integraciones. Además, se desarrollarán pruebas automatizadas para cada nueva integración y se establecerá un protocolo de respuesta rápida para resolver fallos críticos en menos de 24 horas.

