\section{Análisis de riesgo}

En el proceso de creación de una organización, resulta fundamental incluir la evaluación de riesgos en las fases de gestión y planificación. Esta evaluación es clave, ya que permite identificar posibles riesgos de diversas naturalezas que podrían afectar negativamente a la organización o a los cuales esta podría estar expuesta. Además, es esencial para diseñar estrategias efectivas que minimicen esos riesgos o reduzcan la probabilidad de que generen efectos adversos. \cite{PMI}.

La Evaluación de Riesgos requiere un análisis exhaustivo de la gravedad y la naturaleza de los posibles impactos negativos que podría enfrentar una nueva propuesta. Asimismo, es crucial identificar métodos eficaces para mitigar los riesgos detectados y considerar alternativas viables a la propuesta original.

\import{./}{Content/12.AnalisisDeRiesgo/FactoresLimitantesObstaculos}

\import{./}{Content/12.AnalisisDeRiesgo/FactoresClaveExito}

\import{./}{Content/12.AnalisisDeRiesgo/RiesgosLegales}

\import{./}{Content/12.AnalisisDeRiesgo/RiesgosOperacionales}
