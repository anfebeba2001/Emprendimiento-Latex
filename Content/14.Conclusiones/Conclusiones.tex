\section{Conclusiones}

% Explicación de porque retorno es relativamente bajo 
El retorno del proyecto está estimado en un retorno a tres años, lo cual puede parecer bajo en comparación con otros emprendimientos tecnológicos. Sin embargo, de esta forma se asegura que el proyecto se consolide exitosamente teniendo contemplado un capital destinado a cubrimiento de riesgos como pueden ser fallos de último momento o cambios inesperados en el mercado. Aparte de esto se tiene en cuenta que es un proyecto que emplea pocos recursos humanos y por ende son recursos que son mas fáciles de evaluar, mantener y gestionar. 

% Justificación del proyecto basado en análisis de DOFA y PEYEA 
El proyecto surge tras analizar la matriz DOFA, identificando como fortaleza la creciente comunidad de nómadas digitales en Bogotá y la oportunidad de ofrecerles una plataforma centralizada de recomendaciones personalizadas. Se reconocen amenazas como la competencia de plataformas globales y debilidades en la penetración inicial del mercado local. La matriz PEYEA sugiere una estrategia de diferenciación enfocada en la personalización y la confianza generada por las reseñas de usuarios reales.

% Explicación de la plataforma basada en el plan de negocio
La aplicación está diseñada para nómadas digitales que buscan estadías, restaurantes, espacios de coworking y otros servicios en Bogotá. Utiliza un sistema de recomendaciones personalizadas basado en las preferencias y reseñas de los propios usuarios, lo que permite una experiencia adaptada a cada perfil. El modelo de negocio contempla alianzas con proveedores locales y la monetización a través de membresías premium y comisiones por reservas.

% Metodologías empleadas
Se emplearon metodologías ágiles para el desarrollo del producto mínimo viable (PMV), permitiendo iteraciones rápidas y validación continua con usuarios reales. Además, se aplicaron técnicas de estudio de mercado para entender las necesidades del usuario objetivo y construir una solución centrada en sus expectativas.

% Justificación financiera basada en el análisis financiero
El análisis financiero muestra que el proyecto es viable, con un bajo costo inicial gracias al desarrollo incremental y la posibilidad de escalar la plataforma. Se proyecta que habrá rentabilidad a partir del tercer año, apalancada en el crecimiento del turismo digital y la economía colaborativa.

% Explicación de muestra de valor del PMV
El prototipo mínimo viable suple las necesidades de los usuarios, permitiendo contribuir a la comunidad de nómadas digitales en Bogotá y recibir recomendaciones personalizadas. Con esta propuesta de valor se tiene una gran entrada en el mercado turístico y se garantiza la viabilidad del proyecto, permitiendo impulsar la plataforma como un referente para la comunidad de nómadas digitales y como un apoyo al sector turístico de la ciudad.
