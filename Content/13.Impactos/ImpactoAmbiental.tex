\subsection{Impacto Ambiental}
En un principio se propone un enfoque digital con la finalidad de disminuir la huella ambiental del proyecto, se fomentarán experiencias turisticas de bajo impacto como rutas caminables, cowokings verdes y alojamientos eco-certificados, que garanticen la seguridad y bienestar de los usuarios, al mismo tiempo que se promueve un estilo de vida saludable y activo.

Como parte del plan a mediano plazo se busca integrar criterios de sosstenibilidad en el algoritmo de recomendaciones, priorizando negocios locales que implementen prácticas sostenibles y responsables con el medio ambiente, además se lanzará un programa de compensación colaborativa donde cada reserva generará un aporte a proyectos de conservación ambiental.

Por ultimo se tiene como objetivo a largo plazo convertirse en un referente de turismo nomada sostenible, mediante un sistema de "Eco-score" que evalúe y promueva prácticas responsables entre los usuarios, incentivando un estilo de vida que respete y cuide el medio ambiente.
