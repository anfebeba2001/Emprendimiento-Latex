\color{red}

\subsection{Impacto ambiental}

Inicialmente, el proyecto buscará minimizar su huella ambiental operando digitalmente y fomentando actividades físicas que no requieren instalaciones energéticamente intensivas. Se incentivarán prácticas de ejercicio al aire libre, que contribuyen al bienestar y la conexión con el medio ambiente.

A medio plazo, se planea integrar criterios de sostenibilidad en todas las operaciones y colaboraciones con negocios locales, buscando promover una cultura de responsabilidad ambiental.

En una visión a largo plazo, la meta es ser un referente en sostenibilidad dentro del sector del fitness, impulsando iniciativas que promuevan un impacto ambiental positivo, como eventos de fitness que incluyan actividades de reforestación o limpieza de espacios naturales.