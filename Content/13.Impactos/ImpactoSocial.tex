
\color{red}
\subsection{Impacto social}

A corto plazo, el proyecto tiene como objetivo mejorar la calidad de vida de los nómadas digitales en Bogotá, ofreciendo servicios de fitness personalizados y actividades físicas que promueven la salud y el bienestar. Se espera fomentar la inclusión social y el bienestar mental a través del ejercicio y la interacción comunitaria.

En el mediano plazo, se prevé ampliar el alcance del proyecto para incluir a otros segmentos de la población local, contribuyendo así a la reducción del sedentarismo y promoviendo estilos de vida activos y saludables en una mayor porción de la sociedad.

A largo plazo, el proyecto aspira a expandirse más allá de Bogotá, extendiendo sus beneficios sociales a otras ciudades y regiones. Esta expansión podría incluir programas adaptados a diversas culturas y necesidades, apoyando así la creación de comunidades más activas y saludables en toda Colombia y posiblemente en otros países.