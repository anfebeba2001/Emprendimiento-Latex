\subsection{Impacto social}

A corto plazo, el proyecto mejorará la integración social de los nómadas digitales en Bogotá, mediante comunidades virtuales que faciliten conexiones auténticas con locales y otros nómadas. Se fomentará la participación en actividades comunitarias y eventos presenciales, promoviendo un sentido de pertenencia y colaboración, además de combatir el aislamiento social que a menudo enfrentan los nómadas digitales.

A mediano plazo, se fortalecerá el tejido social mediante la creación de redes de apoyo y colaboración entre nómadas digitales y residentes locales. Se implementarán programas de voluntariado y actividades conjuntas que fomenten el intercambio cultural y la inclusión social, contribuyendo a una comunidad más cohesionada y diversa.

A largo plazo, el proyecto aspira a transformarse en un puente global de cohesión social mediante la creación de "hubs de convivencia sostenible" en diversas ciudades del mundo. Estos espacios integrarán a nómadas digitales en proyectos colaborativos que aborden desafíos sociales y ambientales locales, promoviendo un modelo de vida nómada que respete y enriquezca las comunidades anfitrionas. Se espera que esta iniciativa inspire a otras ciudades a adoptar enfoques similares, fomentando una cultura global de colaboración y sostenibilidad, sentando las bases para comunidades interculturales resilientes.
