\subsection{Impacto económico}

En primer lugar, el proyecto generará empleo directo en áreas tecnológicas y de oprecaiones, mientras dinamiza economías locales atrayendo nómadas digitales a destinos emergentes como Bogotá.Esto incrementa el consumo en servicios como alojamientos,coworkings, restaurantes y transporte, beneficiando a la comunidad local.

A mediano plazo, se fortalecerán cadenas de valor turístico mediante alianzas con proveedores locales para ofertas integradas experiencias auténticas.Se lanzarán programas de formación y capacitación para emprendedores locales, fomentando el desarrollo de habilidades en el sector turístico y tecnológico.

Manteniendo el enfoque a largo plazo, la plataforma se consolidará como motor de desarrollo económico sostenible mediante su expansión a otras ciudades y países con hubs regionales que contrate talento local y promueva el emprendimiento en el sector del turismo nómada digital. Se creará un "Ecosistema Nómada" para incubar startips de turismo tech.
