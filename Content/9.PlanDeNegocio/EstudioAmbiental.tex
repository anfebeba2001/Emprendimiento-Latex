\subsection{Estudio Ambiental}

En Colombia, el desarrollo de software está regulado principalmente por la Ley 23 de 1982 y la Ley 603 del 2000. La ley 23 de 1982 establece el marco legal para los derechos de autor, considerando el software como una obra literaria, mientras la Ley 603 del 2000, también conocida como la "Ley del software", se enfoca en la protección, desarrollo y comercialización de programas informáticos, incluyendo la obligación de declarar el cumplimiento de las normas de software en los informes de gestión.

Así mismo se tendrá en cuenta para el desarrollo del proyecto la ley 1581 de 2012, la cual establece las disposiciones generales para la protección de datos personales, incluyendo los derechos de los titulares, las obligaciones de los responsables y administradores del tratamiento de dato, y los mecanismos de protección y sanación.

Por consiguiente el proyecto se desarrollará bajo las normativas de las leyes anteriormente mencionadas, con el compromiso de protección de datos además de la correcta declaración del software desarrollado como una obra literaria.

\subsection*{Impactos amientales}

Triavel como proyecto enfocado en el desarrollo de software para recomendaciones seguras a nómadas digitales, reconoce los impactos ambientales asociados a sus operaciones tecnológicas y actividades complementarias.Se compromete a evaluar y gestionar responsablemente los efectos derivados de el cicblo completo de nuestro producto, tal como el diseño, ejecución, operación y actualizaciones. Además de planificar de manera consciente los recursos tecnológicos utilizados tal como la infraestructura digital y dispositivos.

\subsection*{Políticas ambientales}
Triavel, en cumplimiento del Decreto 1076 de 2015 (SINA) y la Ley 1715 de 2014, adopta las siguientes acciones para minimizar su impacto ambiental:

\begin{itemize}
    \item Energía limpia: Nuestra infraestructura tecnológica opera con proveedores cloud que usan energías renovables, acogiéndonos a los incentivos tributarios de la Ley 1715.
    \item Gestión de residuos: Los equipos electrónicos obsoletos son gestionados a través de gestores autorizados, siguiendo el Decreto 284 de 2018 sobre RAEE.
    \item Compensación ambiental: Medimos nuestra huella de carbono con la Resolución 40807 de 2018 y compensamos mediante apoyo a proyectos de conservación en ecosistemas estratégicos colombianos.
\end{itemize}

Triavel integra la sostenibilidad en su core tecnológico: nuestra app promueve rutas seguras y eco-amigables, contribuyendo a un turismo responsable.
