\subsection{Estudio Ambiental}

{\color{red}
Según el decreto 1360 de 1989, en su artículo 1, se establece que el software se considera una forma de expresión literaria, conforme a lo estipulado en la ley 23 de 1982 sobre Derechos de Autor. Por consiguiente, cualquier software que se desarrolle debe ser registrado como una obra literaria, proceso que será facilitado por la dirección nacional de derechos de autor. 

El proyecto también debe cumplir con las disposiciones establecidas en la Ley 1581 de 2012 de Protección de Datos Personales. Esto implica que cualquier información obtenida por los clientes debe ser tratada de manera transparente y segura, obteniendo el consentimiento informado de los individuos. Adicionalmente, se debe garantizar que los datos serán utilizados para fines específicos y que los usuarios tengan la posibilidad de rectificar, actualizar o suspender el uso de sus datos.  

%No estoy del todo segura qué tiene que ver esta parte con el estudio ambiental, pareciera más del estudio legal. (si no corresponde, hay que pasarlo a esa sección)

\subsubsection*{Impactos ambientales}

 ActiveNomads S.A.S se especializa en el desarrollo de software reconoce los aspectos y efectos resultantes de sus operaciones, funciones y recursos utilizados en su entorno. Además, se considerarán las diferentes etapas de ejecución, operación, mantenimiento e implementación de proyectos. 

Cualquier tipo de impacto ambiental identificado se aborará mediante medidas de mitigación ambiental que garanticen la presentación del entorno por medio de acciones preventivas, de control, mitigación, restauración y compensación de posibles impactos negativos. 

\subsubsection*{Políticas ambientales}

ActiveNomads S.A.S se adherirá a la Norma Internacional NTC ISO 14001:2015, la cual define los criterios necesarios para que una organización administre la prevención de la contaminación y controle las actividades, procesos y productos que puedan ocasionar un impacto adverso en el medio ambiente. Esto reflejará su coherencia con el compromiso fundamental de proteger y respetar el entorno natural. 

En virtud de los lineamientos establecidos por esta norma internacional, ActiveNomads S.A.S asume los siguientes compromisos: 

\begin{itemize}
    \item Implementar, mantener y mejorar un Sistema de Gestión Ambiental. 
    
    \item Establecer controles operativos para prevenir o minimizar su impacto negativo en el medio ambiente. 
    
    \item Proporcionar formación y sensibilización ambiental al personal de la organización.
    
    \item Promover el uso eficiente de los recursos naturales como el agua, la energía y los materiales mediante la implementación de prácticas de conservación y adopción de tecnología sostenibles. 
    
    \item Fomentar actividades ecológicas bajo el amparo de la empresa en las que se pueden tener, salidas de limpieza en el parque o jornadas para plantar árboles.
    
\end{itemize}
}