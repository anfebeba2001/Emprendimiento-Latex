\subsection{Estudio Ambiental}

El desarrollo de software en Colombia está regulado por la Ley 23 de 1982 (derechos de autor), Ley 603 del 2000 (protección y comercialización de software) y la Ley 1581 de 2012 (protección de datos personales). El proyecto cumplirá estas normativas, garantizando la protección de datos y la correcta declaración del software.

\subsection*{Impactos ambientales}

Triavel reconoce los impactos ambientales de sus operaciones tecnológicas y se compromete a gestionarlos responsablemente durante todo el ciclo de vida del producto, planificando el uso de recursos tecnológicos e infraestructura digital.

\subsection*{Políticas ambientales}

Triavel cumple el Decreto 1076 de 2015 (SINA) y la Ley 1715 de 2014, adoptando acciones como:
\begin{itemize}
    \item Uso de energía renovable en proveedores cloud (Ley 1715).
    \item Gestión de residuos electrónicos según el Decreto 284 de 2018 (RAEE).
    \item Medición y compensación de huella de carbono (Resolución 40807 de 2018) mediante apoyo a proyectos de conservación.
\end{itemize}
La app promueve rutas seguras y eco-amigables, fomentando el turismo responsable.
