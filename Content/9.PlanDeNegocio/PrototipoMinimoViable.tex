\subsection{Prototipo mínimo viable}
{\color{red}
El objetivo es desarrollar una aplicación que ofrezca servicios de bienestar y entrenamiento personalizado para nómadas digitales en Bogotá. Para ello, se plantea el uso de una serie de tecnologías y arquitecturas que permitan la creación de una plataforma escalable, flexible y adaptada a las necesidades de los usuarios, maximizando su experiencia a través de soluciones de vanguardia en fitness y bienestar.



\textbf{Tecnologías}

\begin{itemize}
    
    \item \textbf{Firebase: } Firebase es una plataforma de desarrollo de aplicaciones web y móviles que incluye servicios de base de datos en la nube, entre otros. Con Firebase, puedes almacenar y sincronizar datos en tiempo real a través de su base de datos en la nube, Firebase Firestore, que permite a los desarrolladores acceder y gestionar datos de manera sencilla desde cualquier lugar.
    
    Además, Firebase elimina la necesidad de configurar y mantener servidores o entornos de base de datos, lo que agiliza el desarrollo al ofrecer una solución gestionada completamente desde su plataforma. Esta será la herramienta que se utilizará para la base de datos del proyecto.

   
    \vspace{2mm}
        \begin{minipage}{0.9\textwidth}
        \centering
        \captionof{figure}[{Logo FireBase Firestore}]{ Logo FireBase Firestore  }
        \label{Firestore}
         \includegraphics[width=0.3\textwidth]{Content/Images/Escudo_UD.png}
        \footnote{Nota. \textup{Fuente: firebase.com}}
    \end{minipage}

    \item \textbf{Node js: }  Node.js, es un entorno en tiempo de ejecución multiplataforma para la capa del servidor (en el lado del servidor) basado en JavaScript. Node.js es un entorno controlado por eventos diseñado para crear aplicaciones escalables, permitiéndote establecer y gestionar múltiples conexiones al mismo tiempo. 
   
    \vspace{2mm}
        \begin{minipage}{0.9\textwidth}
        \centering
        \captionof{figure}[{Logo Node.js}]{ Logo Node.js  }
        \label{nodeJS}
        \includegraphics[width=0.3\textwidth]{Content/Images/Escudo_UD.png}
          %esto es lo nuevo que agregue
        \footnote{}{Nota. \textup{Fuente: NODEJS.com}}
    \end{minipage}
    
      \item \textbf{Angular: } Angular es un framework de desarrollo web creado por Google que facilita la construcción de aplicaciones web dinámicas y robustas. Lo bueno de Angular es que te permite estructurar y organizar el código de manera eficiente, haciendo que sea mucho más fácil mantener y ampliar proyectos grandes.
    
    Angular ofrece un enfoque basado en componentes reutilizables, lo que ayuda a que las aplicaciones sean más fáciles de desarrollar y mantener. Además, incluye herramientas integradas para el manejo de formularios, peticiones HTTP, enrutamiento, entre otros, sin la necesidad de instalar librerías adicionales. Angular también puede ejecutarse tanto en el lado del cliente como en el servidor, lo que le da flexibilidad para adaptarse a diferentes tipos de proyectos.
    
    \vspace{2mm}
        \begin{minipage}{0.9\textwidth}
        \centering
        \captionof{figure}[{Logo Angular}]{Logo Angular}
        \label{angularLogo}
        \includegraphics[width=0.3\textwidth]{Content/Images/Escudo_UD.png}
        \footnote{Nota. \textup{Fuente: angular.io}}
    \end{minipage}
    
\end{itemize}
\vspace{5mm}
\textbf{Arquitectura} 
\begin{itemize}
    
    \item \textbf{Arquitectura de microservicios: } La arquitectura de microservicios es un enfoque de desarrollo de aplicaciones que divide el software en pequeños servicios independientes, cada uno enfocado en realizar una función específica del negocio. Estos microservicios funcionan de manera autónoma y pueden estar desarrollados en diferentes lenguajes de programación. Se comunican entre sí a través de APIs y, en muchos casos, utilizan su propio sistema de almacenamiento, lo que contribuye a la estabilidad y escalabilidad de la aplicación al evitar sobrecargas y puntos únicos de fallo.
    
    Este enfoque ha sido seleccionado para la arquitectura del back-end, ya que facilita la implementación modular de los servicios necesarios para el negocio. Además, permite que cada servicio se despliegue de manera independiente, lo cual es ideal para su posterior despliegue en plataformas como Firebase.
    
    \vspace{2mm}
    \begin{minipage}{0.9\textwidth}
        \centering
        \captionof{figure}[{Arquitectura de Microservicios.}]{Ejemplo de arquitectura de microservicios.}
        \label{microservicios}
        \includegraphics[width=0.3\textwidth]{Content/Images/Escudo_UD.png}
        \footnote{Nota. \textup{Fuente: decidesoluciones.es/arquitectura-de-microservicios}}
    \end{minipage}
    
    \item \textbf{Arquitectura hexagonal: } La Arquitectura Hexagonal, también conocida como Arquitectura de Puertos y Adaptadores, es un patrón de diseño propuesto por Alistair Cockburn, que organiza el software en capas con el objetivo de aislar cada funcionalidad, permitiendo una evolución más ordenada y modular del sistema. El núcleo de esta arquitectura es el dominio, y cada lado del hexágono representa una interacción con servicios externos, facilitando así la integración y adaptabilidad a futuros cambios.
    
    Esta arquitectura permite que los componentes del sistema se mantengan independientes unos de otros, facilitando el mantenimiento y la escalabilidad. En este caso, el dominio principal se mantiene aislado mientras que los adaptadores interactúan con bases de datos, APIs, y otros servicios externos.
    
    \vspace{2mm}
    \begin{minipage}{0.9\textwidth}
        \centering
        \captionof{figure}[{Arquitectura Hexagonal.}]{Ejemplo de arquitectura hexagonal.}
        \label{hexagonal}
        \includegraphics[width=0.3\textwidth]{Content/Images/Escudo_UD.png}
        \footnote{Nota. \textup{Fuente: https://softwarecrafters.io/react/arquitectura-hexagonal-frontend}}
    \end{minipage}

    
\end{itemize}

\textbf{Infraestructura }
\newline
La computación en la nube sigue siendo un componente fundamental de este proyecto, permitiendo la entrega y uso de almacenamiento, servidores, aplicaciones y otros recursos a través de Internet, bajo un modelo de pago por consumo. En lugar de depender de centros de datos locales o infraestructuras físicas, la infraestructura utilizada se apoya en servicios remotos y gestionados, lo que permite una escalabilidad eficiente y una gestión optimizada de los recursos.

Para este proyecto, se ha seleccionado Firebase como plataforma en la nube. Firebase ofrece una infraestructura integral que facilita el desarrollo y despliegue de aplicaciones, proporcionando servicios como bases de datos en tiempo real, almacenamiento, autenticación y hosting, todos gestionados de manera centralizada. Esto elimina la necesidad de administrar servidores físicos o realizar configuraciones complejas, ya que la plataforma maneja estos aspectos, permitiendo que el desarrollo se enfoque exclusivamente en la creación de la aplicación. \cite{FirebaseCloud}.
\newline
\textbf{Software as a Service (SaaS) :}
El modelo SaaS (Software como Servicio) se utiliza de manera prominente en este proyecto, ya que Firebase funciona bajo este esquema. Firebase, al ser un servicio gestionado, permite a los desarrolladores centrarse en el desarrollo del software sin tener que preocuparse por la administración de servidores, las actualizaciones de seguridad o la gestión de licencias. Los usuarios finales pueden acceder a la aplicación y sus funcionalidades desde cualquier ubicación con una conexión a Internet, lo que simplifica tanto el uso como la operación del sistema. \cite{FirebaseSAAS}


\textbf{Requerimientos}

\begin{enumerate}
    \item Funcionales
        \par\vspace{2mm}
        \begin{minipage}{0.9\textwidth}
        \centering
        \captionof{table}[{Requerimiento funcional RF-001.}]{ Requerimiento funcional RF-001. }
        \label{req1}
        \includegraphics[width=0.3\textwidth]{Content/Images/Escudo_UD.png}
        \footnote{}{Nota. \textup{Fuente : Autores.}}
        \end{minipage}
        
        \vspace{2mm}
        \begin{minipage}{0.9\textwidth}
        \centering
        \captionof{table}[{Requerimiento funcional RF-002.}]{ Requerimiento funcional RF-002. }
        \label{req2}
        \includegraphics[width=0.3\textwidth]{Content/Images/Escudo_UD.png}
        \footnote{}{Nota. \textup{Fuente : Autores.}}
        \end{minipage}
        
        \vspace{2mm}
        \begin{minipage}{0.9\textwidth}
        \centering
        \captionof{table}[{Requerimiento funcional RF-003.}]{ Requerimiento funcional RF-003. }
        \label{req3}
        \includegraphics[width=0.3\textwidth]{Content/Images/Escudo_UD.png}
        \footnote{}{Nota. \textup{Fuente : Autores.}}
        \end{minipage}
        
        \vspace{2mm}
        \begin{minipage}{0.9\textwidth}
        \centering
        \captionof{table}[{Requerimiento funcional RF-004.}]{ Requerimiento funcional RF-004. }
        \label{req4}
        \includegraphics[width=0.3\textwidth]{Content/Images/Escudo_UD.png}
        \footnote{}{Nota. \textup{Fuente : Autores.}}
        \end{minipage}
        
        \vspace{2mm}
        \begin{minipage}{0.9\textwidth}
        \centering
        \captionof{table}[{Requerimiento funcional RF-005.}]{ Requerimiento funcional RF-005. }
        \label{req5}
        \includegraphics[width=0.3\textwidth]{Content/Images/Escudo_UD.png}
        \footnote{Nota. \textup{Fuente : Autores.}}
        \end{minipage}
        
        \vspace{2mm}
    %    \begin{minipage}{0.9\textwidth}
    %    \centering
    %    \captionof{table}[{Requerimiento funcional RF-006.}]{ Requerimiento funcional RF-006. }
    %    \label{req6}

    %    \end{minipage}
        

        
    \item No funcionales
    
        
        \vspace{2mm}
        \begin{minipage}{0.9\textwidth}
        \centering
        \captionof{table}[{Requerimiento no funcional RNF-001.}]{ Requerimiento no funcional RNF-001. }
        \label{reqnf1}
         \includegraphics[width=1\textwidth]{Content/Images/Escudo_UD.png}
        \footnote{Nota. \textup{Fuente : Autores.}}
        \end{minipage}
        
        \vspace{2mm}
        \begin{minipage}{0.9\textwidth}
        \centering
        \captionof{table}[{Requerimiento no funcional RNF-002.}]{ Requerimiento no funcional RNF-002. }
        \label{reqnf2}
        \includegraphics[width=1\textwidth]{Content/Images/Escudo_UD.png}
        \footnote{Nota. \textup{Fuente : Autores.}}
        \end{minipage}

        \vspace{2mm}
        \begin{minipage}{0.9\textwidth}
        \centering
        \captionof{table}[{Requerimiento no funcional RNF-003.}]{ Requerimiento no funcional RNF-003. }
        \label{reqnf3}
        \includegraphics[width=1\textwidth]{Content/Images/Escudo_UD.png}
        \footnote{}{Nota. \textup{Fuente : Autores.}}
        \end{minipage}
        
\end{enumerate}

\textbf{Casos de uso}
    
    \vspace{2mm}
    \begin{minipage}{0.9\textwidth}
    \centering
    \captionof{figure}[{Caso de uso RF-001.}]{ Caso de uso RF-001. }
    \label{caso1}
    \includegraphics[width=0.3\textwidth]{Content/Images/Escudo_UD.png}
    \footnote{Nota. \textup{Fuente : Autores.}}
    \end{minipage}
    
    \vspace{2mm}
    \begin{minipage}{0.9\textwidth}
    \centering
    \captionof{figure}[{Caso de uso RF-002.}]{ Caso de uso RF-002. }
    \label{caso2}
    \includegraphics[width=0.3\textwidth]{Content/Images/Escudo_UD.png}
    \footnote{Nota. \textup{Fuente : Autores.}}
    \end{minipage}
    
    \vspace{2mm}
    \begin{minipage}{0.9\textwidth}
    \centering
    \captionof{figure}[{Caso de uso RF-003.}]{ Caso de uso RF-003. }
    \label{caso3}
    \includegraphics[width=0.3\textwidth]{Content/Images/Escudo_UD.png}
    \footnote{Nota. \textup{Fuente : Autores.}}
    \end{minipage}
    
    \vspace{2mm}
    \begin{minipage}{0.9\textwidth}
    \centering
    \captionof{figure}[{Caso de uso RF-004.}]{ Caso de uso RF-004. }
    \label{caso4}
    \includegraphics[width=0.3\textwidth]{Content/Images/Escudo_UD.png}
    \footnote{Nota. \textup{Fuente : Autores.}}
    \end{minipage}
    
    \vspace{2mm}
    \begin{minipage}{0.9\textwidth}
    \centering
    \captionof{figure}[{Caso de uso RF-005.}]{ Caso de uso RF-005. }
    \label{caso1}
    \includegraphics[width=0.3\textwidth]{Content/Images/Escudo_UD.png}
    \footnote{Nota. \textup{Fuente : Autores.}}
    \end{minipage}
    


\textbf{Vistas principales del prototipo}
De acuerdo a los requerimientos funcionales y no funcionales junto con los casos de uso, a continuación, en las figuras \ref{prot1}, \ref{prot2}, \ref{prot3}, \ref{prot4}, \ref{prot5}, \ref{prot6}, \ref{prot7}, \ref{prot8}, se adjuntan las vistas que representan el núcleo del plan de negocio. Todas las ilustraciones utilizadas en la creación de este prototipo fueron generadas y desarrolladas específicamente para el presente proyecto.

Los manuales para la instalación y el uso de este prototipo se pueden encontrar en los anexos del documento:
\begin{itemize}
    \item \hyperref[manual-usuario]{Manual de Usuario}
    \item \hyperref[manual-instalacion]{Manual de Instalación}
    \item \hyperref[manual-despliegue]{Manual de Despliegue en Vercel}
\end{itemize}


    \vspace{2mm}
    \begin{minipage}{0.9\textwidth}
    \centering
    \captionof{figure}[{Vista principal}]{ Vista principal }
    \label{prot1}
    \includegraphics[width=0.3\textwidth]{Content/Images/Escudo_UD.png}
    \footnote{Nota. \textup{Fuente: Autores}}
    \end{minipage}
    
      \vspace{2mm}
    \begin{minipage}{0.9\textwidth}
    \centering
    \captionof{figure}[{Vista nosotros}]{ Vista nosotros }
    \label{prot2}
    \includegraphics[width=0.3\textwidth]{Content/Images/Escudo_UD.png}
    \footnote{Nota. \textup{Fuente: Autores}}
    \end{minipage}
    
    \vspace{2mm}
    \begin{minipage}{0.9\textwidth}
    \centering
    \captionof{figure}[{Vista registro}]{ Vista registro }
    \label{prot3}
    \includegraphics[width=0.3\textwidth]{Content/Images/Escudo_UD.png}
    \footnote{Nota. \textup{Fuente: Autores}}
    \end{minipage}
    
    \vspace{2mm}
    \begin{minipage}{0.9\textwidth}
    \centering
    \captionof{figure}[{Vista registro}]{ Vista registro }
    \label{prot4}
    \includegraphics[width=0.3\textwidth]{Content/Images/Escudo_UD.png}
    \footnote{Nota. \textup{Fuente: Autores}}
    \end{minipage}
    
    \vspace{2mm}
    \begin{minipage}{0.9\textwidth}
    \centering
    \captionof{figure}[{Vista mi progreso}]{ Vista mi progreso }
    \label{prot5}
    \includegraphics[width=0.3\textwidth]{Content/Images/Escudo_UD.png}
    \footnote{Nota. \textup{Fuente: Autores}}
    \end{minipage}
    
        \vspace{2mm}
    \begin{minipage}{0.9\textwidth}
    \centering
    \captionof{figure}[{Vista eventos}]{ Vista eventos }
    \label{prot6}
    \includegraphics[width=0.3\textwidth]{Content/Images/Escudo_UD.png}
    \footnote{Nota. \textup{Fuente: Autores}}
    \end{minipage}

        \vspace{2mm}
    \begin{minipage}{0.9\textwidth}
    \centering
    \captionof{figure}[{Vista gimnasios}]{ Vista gimnasios }
    \label{prot7}
    \includegraphics[width=0.3\textwidth]{Content/Images/Escudo_UD.png}
    \footnote{Nota. \textup{Fuente: Autores}}
    \end{minipage}

    \vspace{2mm}
    \begin{minipage}{0.9\textwidth}
    \centering
    \captionof{figure}[{Vista mis datos}]{ Vista mis datos }
    \label{prot8}
    \includegraphics[width=0.3\textwidth]{Content/Images/Escudo_UD.png}
    \footnote{Nota. \textup{Fuente: Autores}}
    \end{minipage}


    
    
    

}