\subsection{Stakeholders}

{\color{red}
Los stakeholders son las personas, grupos, organizaciones o entidades que tienen un interés o se ven afectados directa o indirectamente por el plan de negocio propuesto, como se evidencia a continuación: 

\begin{itemize}
    \item Clientes o usuarios finales: Personas con interés en salud y fitness que buscan mejorar su condición física o alcanzar objetivos específicos mediante el uso de la herramienta y profesionales ocupados que necesitan flexibilidad horaria. 

    Entre los usuarios meta se encuentran miembros de comunidades fitness como SmartFit o Bodytech y usuarios de gimansios locales en Bogotá.
    
    \item Entrenadores o profesionales del deporte: Entrenadores certificados en disciplinas específicas (pesas, cardio, funcional, etc.), nutricionistas registrados en COLNUD (Colegio Colombiano de Nutricionistas y Dietistas) y fisioterapeutas con experiencia en recuperación deportiva. 

    Los aliados potenciales son entrenadores que trabajan de forma independiente o en cadenas como Bodytech, SmartFit y Action Fitness. 
    
    \item Equipo interno de la empresa: Desarrolladores de software y desarrolladores UX/UI que crean y mantengan la plataforma digital con interfaces amigables que motiven al uso, además de soporte al cliente para la resolución de dudas sobre la plataforma o planes. 

    Pueden ser contratados por medio de plataformas de desarrollo como Workana o Torre o con agencias como que ofrezcan servicios tecnológicos y consultoría. 
    
    \item Socios estratégicos: Centros deportivos o gimnasios locales para promover la plataforma entre sus clientes, así como empresas de tecnología que ofrezcan herramientas de videollamadas y análisis de datos. 

    Entre los socios estratégicos se plantea la promoción cruzada con cadenas como Bodytech, SmartFit, gimnasios locales y pequeñas y medianas empresas que comercien artículos deportivos o suplementos alimenticios.     
    
    \item Inversionistas o financiadores: Financiamiento externo por entidades de crédito o bancarias e inversionistas que aportan capital inicial al desarrollo de la plataforma.     
    
    \item Proveedores de tecnología: Proveedores de hosting y mantenimiento de la plataforma web y plataformas de pago en línea como intermediarios para manejar los pagos de suscripciones. 

    Puede implementarse los servicios de AWS (Amazon Web Services) o Azure en cuanto al almacenamiento y hosting y pasarelas de pago como Mercado Pago, PayU o Epayco para transacciones locales. 

     \item Comunidad de Bogotá: Grupos locales de fitness que puedan promover la plataforma y organizaciones de bienestar que recomienden el servicio como parte de sus programas. 

     Entre las comunidades disponibles se encuentran grupos como Run Bogotá o CrossFit Bogotá, así como alianzas con programas locales como Ciclovía para afianzar la promoción del sitio. 

     \item Entidades gubernamentales y regulatorias: Instituciones de salud y deporte para asegurar que los servicios cumplen con normativas locales y autoridades fiscales y regulatorias para cumplir con los requisitos legales y tributarios. 

     Entre las entidades que respalden el proyecto, se encuentra el Ministerio de Salud y Proyección Social de Colombia, así como la Dirección de Impuestos y Aduanas Nacionales (DIAN) para el cumplimiento de obligaciones fiscales, en cuando al apoyo de la promoción de actividades físicas, el Instituto Distrital de Recreación y Deporte (IDRD). 

\end{itemize}


}