\subsection{Stakeholders}
Stakeholders son los grupos o individuos que tienen un interés (directo o indirecto) en una empresa, ya que pueden verse afectados por sus actividades o tener la capacidad de influir en su éxito. En el caso de Triavel, los stakeholders incluyen: 
\begin{itemize}
\item Clientes o usuarios finales: Nómadas digitales (remotos, freelancers, empleados internacionales) que visitan Bogotá y priorizan seguridad en sus desplazamientos, así como viajeros de negocios que necesitan recomendaciones validadas en tiempo real.

Entre los usuarios meta se encuentran miembros de comunidades como Digital Nomads Bogotá, coworkings destacados (Ej.: WeWork, Selina), y huéspedes de hostales para nómadas (Ej.: Masaya, Cranky Croc).

\item Aliados estratégicos:

Empresas de tecnología: Plataformas de transporte (Uber, Beat), mapas digitales (Google Maps, Waze) para integraciones API.

Sector turístico: Hostales, restaurantes y coworkings certificados como "seguros" por Triavel (Ej.: Ágora Bogotá, Casa Ánimo).

Entidades públicas: Secretaría de Seguridad de Bogotá y Cámara de Comercio para datos oficiales de criminalidad.

\item Validadores locales: Expertos en seguridad urbana (expertos en georriesgos), guías turísticos certificados y colaboradores freelance que verifican in situ las zonas recomendadas.

Los aliados potenciales incluyen consultores de seguridad como Securitas Colombia y startups de smart cities (Ej.: Quantil, Urbania).

\item Inversores y aceleradoras: Fondos de impacto social (Ej.: Polymath Ventures), programas como Apps.co (MinTIC) y aceleradoras de turismo tecnológico (Ej.: Wayra Hispam).
\item Medios de comunicación y creadores de contenido: Periodistas de turismo, influencers de viajes y editores de guías digitales que difunden información sobre destinos seguros.

Los aliados clave incluyen medios como Bogotá Travel, influencers nómadas reconocidos (ej: Nomad Capitalist) y canales especializados en seguridad para viajeros.

\item Aseguradoras y servicios de protección al viajero: Compañías que ofrecen seguros médicos, contra robos o asistencia en viajes para nómadas digitales.

Entre los socios estratégicos destacan empresas como World Nomads Insurance y Assist Card Colombia, que pueden integrar tus datos de seguridad en sus productos.
\end{itemize}
