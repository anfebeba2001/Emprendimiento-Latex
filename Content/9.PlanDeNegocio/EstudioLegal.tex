\subsection{Estudio legal}
El estudio legal es un análisis exhaustivo de la situación jurídica de una empresa, que busca identificar y evaluar los aspectos legales que pueden afectar su funcionamiento y desarrollo. Este estudio es fundamental para garantizar el cumplimiento normativo y la protección de los derechos e intereses de la empresa, así como para prevenir riesgos legales y financieros.

\subsection*{Tipo de sociedad}
Triavel se constituirá como una Sociedad por Acciones Simplificada (SAS) bajo la Ley 1258 de 2008, al igual que el 54\% de las empresas en Colombia, para aprovechar sus beneficios tributarios, flexibilidad operativa y costos reducidos. Esta estructura permite agilidad en trámites (constitución en 48 horas), no exige revisor fiscal en etapas iniciales y ofrece responsabilidad limitada para los socios, protegiendo su patrimonio personal. Además, facilita la atracción de inversionistas mediante acciones flexibles y adapta sus estatutos a las necesidades del negocio, priorizando escalabilidad internacional y eficiencia administrativa, clave para una startup tecnológica en el sector de seguridad para nómadas digitales.

\textbf{Especificaciones generales}
\newline

La empresa será construida bajo las siguientes pautas:

\begin{itemize}
    \item \textbf{Socios : } Andrés Felipe Bejarano Barón con Cédula de Ciudadanía No ----- y Jeferson David Nieto Gaona con Cédula de Ciudadanía No 1001272402 
    \item \textbf{Nombre de la empresa :} Traivel S.A.S
    \item \textbf{Duración :} Tiempo indefinido.
    \item \textbf{Objeto social :} Empresa cuyo objeto se encuentra en el desarrollo e innovación de herramientas tecnológicas.
    \item \textbf{Responsabilidad sobre los aportes :} 50/50 entre socios.
    \item \textbf{Representante legal: } Andrés Felipe Bejarano Barón
\end{itemize}

\subsection*{Obligaciones legales}
Las obligaciones legales establecidas por la Ley 1258 de 2008 para las SAS incluyen:
\begin{enumerate}
    \item Registro ante la camara de comercio.
    \item Inscripción en el Registro Único Tributario (RUT) ante la DIAN.
    \item Obtención del NIT (Número de Identificación Tributaria).
    \item Registro de libros contables y actas.
    \item Inscripción de los socios y accionistas en el libro de registro de acciones.
    \item Afiliación a la seguridad social de los empleados.
    \item Presentación de declaraciones tributarias (IVA, renta, retención en la fuente).
    \item Cumplimiento de las obligaciones laborales y de seguridad social.
\end{enumerate}

\subsection*{Proceso para disolución de la empresa}
\begin{enumerate}

    \item La disolución de la empresa puede ser voluntaria o forzosa, y debe ser aprobada por la asamblea de accionistas.
    \item Se debe elaborar un acta de disolución y registrarla en la cámara de comercio.
    \item Nombrar un liquidador que se encargue de liquidar los activos y pasivos de la empresa, realizando el inventario correspondiente.
    \item Publicar un aviso de disolución en un medio de comunicación local.
    \item Liquidar los activos y pasivos de la empresa, y distribuir el remanente entre los socios,al pagar las obligaciones se debe priorizar los salarios y las prestaciones sociales, seguido de los impuestos correspondientes a la DIAN.
    \item Presentar la declaración final de renta ante la DIAN.
    \item \item Realizar el acta de liquidación con detalle en activos vendidos, pasivos pagados y remanente distribuido entre socios.
    \item Registrar la liquidación en la cámara de comercio,cancelar la matrícula mercantil y liquidar cuentas bancarias.
    \item Cancelar el registro de los libros contables y actas.
\end{enumerate}

\subsection*{Propiedad intelectual}
La propiedad intelectual es un conjunto de derechos que protegen las creaciones de la mente, como invenciones, obras literarias y artísticas, diseños industriales y marcas. En el caso de Triavel, la propiedad intelectual es fundamental para proteger su software y tecnología desarrollada. Para ello, se recomienda registrar el software ante la Dirección Nacional de Derechos de Autor y considerar la posibilidad de patentar invenciones o diseños industriales relacionados con su producto.Es importante establecer acuerdos de confidencialidad con empleados y colaboradores para proteger información sensible y secretos comerciales,además se debe incluir cláusulas en los estatutos de la SAS que reconozcan el software como un activo intangible de la sociedad, asegurando su titularidad y control legal .
