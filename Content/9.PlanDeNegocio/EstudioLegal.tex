\subsection{Estudio legal}
{\color{red}
El objetivo del estudio legal es evaluar la viabilidad dentro el contexto normartivo establecido por la Constitución Colombiana, identificando las regulaciones y leyes que afectan a la empresa, así como sus alcances,  limitaciones y obligaciones. En resumen, se buscar definir su formalizacón ante las autoridades estatales. 

\subsubsection*{Tipo de sociedad}

ActiveNomads se constituirá  bajo la Sociedad por Acciones Simplificadas (S.A.S), introducida en Colombia en el año 2008 con la ley 1258, rigiéndose por reglas aplicables a sociedades anónimas. 

Esta forma empresarial resulta altamente beneficiosa en términos tributarios, donde aproximadamente el 54\% de las empresas establecidas en Colombia optan por este tipo de sociedad, puesto que facilita a los empleadores simplificar procesos administrativos y dar inicio a proyectos con inversiones iniciales reducidas. 

Adicionalmente, suelen tener tasas reducidas de impuestos y exenciones en algunos trámites, no requieren de contratación de un revisor fiscal y pueden ser constituidas por personas naturales como jurídicas. 

\textbf{Especificaciones generales}
\newline

La empresa será construida bajo las siguientes pautas:

\begin{itemize}
    \item \textbf{Socios : } Andrés David Beltrán Rojas con Cédula de Ciudadanía No 1000802597 y Luisa María Lugo Flórez con Cédula de Ciudadanía No 1000236130
    \item \textbf{Nombre de la empresa :} ActiveNomads S.A.S
    \item \textbf{Duración :} Tiempo indefinido.
    \item \textbf{Objeto social :} Empresa cuyo objeto se encuentra en el desarrollo e innovación de herramientas tecnológicas.
    \item \textbf{Responsabilidad sobre los aportes :} 50/50 entre socios.
    \item \textbf{Representante legal: } Luisa María Lugo Flórez
\end{itemize}

\subsubsection*{Obligaciones legales}

Las obligaciones legales que se detallarán a continuación están enmarcadas dentro de la ley 158 de 2008, que establece la creación de la Sociedad por Acciones Simplificada.

\begin{enumerate}
    \item Asamblea Ordinaria de Accionistas por lo menos una vez al año.
    \item Renovación de la matrícula mercantil dentro de los primeros tres meses del año en la Cámara de Comercio.
    \item Firma electrónica de los representantes legales, resolución 070 de la DIAN.
    \item Agentes de retención en la fuente a título de Renta, Iva, Ica, etc.
    \item Expedición facturas electrónicas.
    \item Gravamen a los movimientos financieros.
    \item Pago el impuesto predial.
    \item Gestión de los libros contables de la empresa.
    \item Tener un revisor fiscal según el monto de sus ingresos o activos.
    \item Cumplimiento de la normativa laboral y ambiental. 
\end{enumerate}

\subsubsection*{Proceso para disolución de la empresa}

Para proceder con la disolución de una empresa constituida bajo la figura de Sociedad por Acciones Simplificadas se seguirán estos pasos:

\begin{enumerate}
    \item Para las S.A.S, Empresas Unipersonales y otras entidades reguladas por la Ley 1014 de 2006, la declaración de disolución por mutuo acuerdo podrá efectuarse a través de un acta o documento privado.
    
    \item Proceder con el registro del acta de disolución en la Cámara de Comercio. Una vez realizado el registro, la sociedad será identificada con el término “en liquidación”. Es necesario entregar una copia del acta en la Cámara de Comercio, la original deberá quedarse en el archivo de la sociedad.
    
    \item El liquidador está obligado a reportar cualquier deuda fiscal de la sociedad a la Oficina de Cobranzas de la DIAN. Se cuenta con un plazo de diez días posteriores al registro de la disolución en la Cámara de Comercio. 
    
    \item Es necesario emitir avisos que informen que la sociedad se encuentra en proceso de liquidación. 
    
    \item El liquidador debe encargarse de elaborar un inventario del patrimonio social y un balance final de la sociedad.
    
    \item El liquidador también es responsable de pagar el pasivo externo de la sociedad y deberá hacer frente a las obligaciones fiscales, además de presentar la declaración final de renta. 
    
    \item Distribución de remanentes entre socios o accionistas por parte del liquidador. 
    
    \item El liquidador es responsable de preparar el proyecto de liquidación, debe incluir como mínimo lo siguiente: 
    
    \begin{itemize}
        \item Inventarios.
        \item Balance general.
        \item Estado de pérdidas y ganancias.
        \item Pasivos de la entidad.
        \item Se solicita el estado de cuenta a la DIAN.
        \item Pago de pasivos.
        \item Indicación del remanente.
        \item Destinación del remanente.
    \end{itemize}
    
    \item Se requiere convocar una reunión de la Junta de Socios o Asamblea de Accionistas con el fin de aprobar el proyecto de liquidación.
    
    \item Al registrar el acta final de la liquidación en la Cámara de Comercio, se debe cancerlar una tarifa de impuesto de registro del 0.7\% sobre el valor de los remanentes de la empresa. Posteriormente, se procederá a pagar su pasivo externo. En caso de no existir remanente a repartir, se considerará como un acto sin cuantía. 
    
\end{enumerate}

\subsubsection*{Propiedad intelectual}

Según el artículo 1 del Decreto 1360 de 1989, se establece: "De conformidad con lo previsto en la Ley 23 de 1982 sobre Derechos de Autor, el soporte lógico (software) se considera como creación propia del dominio literario”. Por lo tanto, cualquier desarrollo de software debe ser sometido a procesos de registro por obra que serán proporcionados por la Dirección Nacional de Derechos de Autor. 
}