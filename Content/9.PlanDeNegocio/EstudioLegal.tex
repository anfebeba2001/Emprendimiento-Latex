\subsection{Estudio legal}
El estudio legal analiza la situación jurídica de la empresa para identificar aspectos legales relevantes, garantizar cumplimiento normativo y prevenir riesgos.

\subsection*{Tipo de sociedad}
Triavel será una Sociedad por Acciones Simplificada (SAS) bajo la Ley 1258 de 2008, como el 54\% de las empresas en Colombia. Esta figura ofrece beneficios tributarios, flexibilidad, costos bajos, agilidad en trámites, responsabilidad limitada y facilita la atracción de inversionistas, ideal para startups tecnológicas.

\textbf{Especificaciones generales}
\newline
\begin{itemize}
    \item \textbf{Socios:} Andrés Felipe Bejarano Barón (CC 1001205352) y Jeferson David Nieto Gaona (CC 1001272402)
    \item \textbf{Nombre:} Traivel S.A.S
    \item \textbf{Duración:} Indefinida
    \item \textbf{Objeto social:} Desarrollo e innovación de herramientas tecnológicas
    \item \textbf{Aportes:} 50/50 entre socios
    \item \textbf{Representante legal:} Andrés Felipe Bejarano Barón
\end{itemize}

\subsection*{Obligaciones legales}
Según la Ley 1258 de 2008, las SAS deben:
\begin{enumerate}
    \item Registrarse en la cámara de comercio
    \item Inscribirse en el RUT ante la DIAN
    \item Obtener NIT
    \item Registrar libros contables y actas
    \item Inscribir socios en el libro de acciones
    \item Afiliar empleados a seguridad social
    \item Presentar declaraciones tributarias (IVA, renta, retención)
    \item Cumplir obligaciones laborales y de seguridad social
\end{enumerate}

\subsection*{Proceso para disolución}
\begin{enumerate}
    \item Disolución aprobada por asamblea de accionistas
    \item Acta de disolución registrada en cámara de comercio
    \item Nombrar liquidador y realizar inventario
    \item Publicar aviso de disolución
    \item Liquidar activos/pasivos y distribuir remanente (priorizando salarios, prestaciones e impuestos)
    \item Declaración final de renta ante DIAN
    \item Acta de liquidación detallada
    \item Registrar liquidación, cancelar matrícula mercantil y cuentas bancarias
    \item Cancelar registro de libros y actas
\end{enumerate}

\subsection*{Propiedad intelectual}
Triavel debe proteger su software y tecnología mediante registro ante la Dirección Nacional de Derechos de Autor, considerar patentes y establecer acuerdos de confidencialidad. Los estatutos deben reconocer el software como activo intangible de la sociedad.
