\subsection{Estudio técnico}

{\color{red}
\subsubsection*{Tamaño}
De acuerdo con la normativa aplicable y considerando las variables de ingresos y costos relacionadas, se clasifica el proyecto dentro del espectro empresarial conforme a la ley colombiana. El tamaño inicial del proyecto se define por su enfoque en el mercado de nómadas digitales en Bogotá, con planes de expansión basados en el análisis de mercado realizado. El crecimiento y desarrollo del proyecto se guiarán por:

\begin{itemize}
    \item \textbf{Recursos Humanos:} El proyecto contará inicialmente con un equipo compacto y multidisciplinario, organizado en áreas clave como administración, marketing, desarrollo de producto. Este equipo se expandirá de forma estratégica conforme el proyecto crezca, asegurando un desempeño eficiente y el logro de objetivos a corto y largo plazo.
    
    \item \textbf{Potencial del Mercado:} El mercado actual muestra una demanda creciente por servicios de bienestar y fitness adaptados a estilos de vida móviles. A través de alianzas estratégicas y una sólida estrategia de marketing, el proyecto apunta a captar significativamente este segmento, logrando una posición destacada en el mercado por la calidad y adaptabilidad de su oferta.
    
    \item \textbf{Factores de Producción:} La plataforma se desarrollará empleando tecnologías de vanguardia que permitan flexibilidad, escalabilidad y una experiencia de usuario excepcional. Se priorizará la adopción de soluciones tecnológicas eficientes que maximicen la relación costo-beneficio, tanto en el desarrollo inicial como en las fases de mantenimiento y actualización.
\end{itemize}

Este enfoque permite establecer una base sólida para el lanzamiento y crecimiento sostenido del proyecto, posicionándolo como un referente en la oferta de servicios de bienestar y fitness para nómadas digitales.

\subsubsection*{Localización}

La localización del proyecto, siendo una plataforma de bienestar y fitness para nómadas digitales, juega un rol crucial no solo en términos de la infraestructura tecnológica requerida sino también en la conexión con el ecosistema local de bienestar en Bogotá. Aunque el servicio es principalmente digital, la base operativa y las colaboraciones locales son esenciales para el lanzamiento y la expansión del servicio. Por tanto, se consideran los siguientes factores para la ubicación física del equipo administrativo y para las actividades presenciales:

\begin{itemize}
    \item \textbf{Accesibilidad:} Ubicación en un área de fácil acceso en Bogotá, tanto para el equipo de trabajo como para la realización de eventos y actividades de bienestar físico.
    \item \textbf{Ambiente Laboral:} Espacios que promuevan un ambiente laboral saludable y productivo, facilitando la creatividad y la colaboración.
    \item \textbf{Costos:} Oficinas y espacios de coworking con costos razonables que se ajusten al presupuesto inicial del proyecto.
    \item \textbf{Infraestructura:} Acceso a servicios esenciales como conexión a internet de alta velocidad, electricidad y servicios sanitarios, fundamentales para el desarrollo tecnológico y la gestión del proyecto.
\end{itemize}

\textbf{Lugares Candidatos:} Para la elección de la localización física del proyecto se han considerado espacios de coworking y centros de bienestar que cumplan con los requisitos mencionados. Estos lugares no solo ofrecen las condiciones ideales para el trabajo colaborativo y la innovación sino que también permiten integrar el proyecto al corazón de la comunidad de bienestar en Bogotá, favoreciendo la creación de alianzas estratégicas y el acceso directo a nuestro público objetivo.

\begin{itemize}
     \item \textbf{ParqueSoft:} Ubicado en el corazón de Bogotá, ParqueSoft se destaca como un ecosistema tecnológico diverso, apoyado por más de 50 entidades especializadas en tecnología. Este hub tecnológico fomenta la sinergia entre empresas emergentes y establecidas, impulsando soluciones innovadoras alineadas con las tendencias digitales más recientes. ParqueSoft se ha consolidado como el cluster más grande de tecnología e innovación en Colombia, jugando un papel fundamental en el apoyo a iniciativas emprendedoras de base tecnológica.

\begin{adjustbox}{center, caption={Precios de Espacios en ParqueSoft}, label={ParqueSoftPrecios}, nofloat=table, vspace={20px}}
    \begin{threeparttable}
        \centering
        \begin{tabular}{|p{7cm}|p{8cm}|}  % Definir las columnas de tamaño adecuado
            \hline
            \cellcolor[HTML]{D9EAD3}\textbf{Tipo de Servicio} & \cellcolor[HTML]{D9EAD3}\textbf{Costo Mensual} \\ \hline
            Suscripción Mensual   & \$96.000 - Total Anual: \$1.152.000 \\ \hline
            Suscripción Semestral & \$78.000/Mes - Total Anual: \$932.000 \\ \hline
            Suscripción Anual     & \$69.000/Mes - Total Anual: \$828.000 \\ \hline
        \end{tabular}
        \begin{tablenotes}[para,flushleft]
            \vspace{2mm}
            \textit{Nota. Fuente: ParqueSoft.com}
        \end{tablenotes}
    \end{threeparttable}
\end{adjustbox}

\item \textbf{Decisión:} Después de evaluar diversas opciones disponibles para establecer la sede del proyecto, ParqueSoft ha sido seleccionado como el lugar ideal debido a su enfoque en la tecnología y el emprendimiento, la accesibilidad de sus tarifas, y el ambiente propicio para el crecimiento y la maduración del proyecto. Los servicios y la infraestructura de ParqueSoft ofrecen el entorno perfecto para fomentar la innovación y facilitar el desarrollo del proyecto, asegurando que se encuentre bien posicionado para su lanzamiento y expansión futura.
\end{itemize}

La selección final del lugar se basará en un análisis detallado de los beneficios, costos y oportunidades que cada opción ofrece, con el objetivo de maximizar el impacto del proyecto tanto en línea como en el entorno local.


\subsubsection*{Tipo de emprendimiento}

El proyecto se clasifica como un Startup, adoptando la Metodología Lean Startup para su desarrollo. Esta elección se basa en el carácter emergente e innovador del producto, el cual se beneficiará de la implementación de tecnologías de la información y promete un crecimiento escalable y sostenible a lo largo del tiempo gracias a un modelo de negocios bien fundamentado.

La metodología "lean startup" se centra en la experimentación rápida y continua. En lugar de dedicar tiempo a elaborar planes de negocio detallados, los emprendedores "lean" prueban sus ideas mediante prototipos mínimos viables (PMV) para aprender de los resultados y adaptarse rápidamente. Este enfoque prioriza el feedback directo de los clientes sobre las suposiciones y predicciones del mercado, instaurando un ciclo iterativo de construir, medir y aprender. Este ciclo busca minimizar el desperdicio de recursos y acelerar el desarrollo de productos o servicios, promoviendo la innovación eficiente tanto en startups como en proyectos empresariales existentes. Permite una respuesta ágil a los cambios del mercado y a las preferencias de los consumidores, constituyendo una estrategia clave para el lanzamiento y la evolución del proyecto.\cite{Blank2013}

\begin{minipage}{0.9\textwidth}
        \centering
        \captionof{figure}[{Metodología Lean Startup}]{Ciclo de Desarrollo Lean Startup}
        \label{leanStartUp}
         \includegraphics[width=0.7\textwidth]{Content/Images/Escudo_UD.png}
        \footnote{Nota. \textup{Fuente: Lean Startup, creando productos escalables \cite{GioSyst3m}}}
\end{minipage}

Mediante este enfoque iterativo, fundamentado en metodologías ágiles, se desarrollará el negocio basándose en hipótesis testadas en el mercado. El PMV permite validar la aceptación del producto con una inversión mínima, determinar si cumple y satisface las necesidades de los clientes y si posee potencial de escalabilidad. Si es necesario, se tomarán acciones para pivotar y ajustar el enfoque, conforme a los aprendizajes obtenidos en el proceso.

\subsubsection*{Equipo de trabajo}
La implementación de este proyecto cuenta con un equipo multidisciplinario compuesto por:

\begin{itemize}
    \item \textbf{Gerente General:} Encargado de la planificación estratégica y la gestión de recursos, definiendo la dirección a corto, medio y largo plazo del proyecto.
    
    \item \textbf{Líder de Ingeniería:} Responsable de guiar el equipo de desarrollo, asegurando la calidad y la innovación en las soluciones tecnológicas ofrecidas.
    
    \item \textbf{Director de I+D+I:} A cargo de impulsar el avance tecnológico e investigativo del proyecto, enfocándose en el desarrollo de nuevas soluciones que contribuyan a la evolución del sector de bienestar y fitness.
\end{itemize}

\begin{adjustbox}{
            center,
            caption=[{Equipo de trabajo.}]{Equipo de trabajo. },
            label={EquipoDeTrabajo},
            nofloat=table, vspace={20px}}
            \resizebox{\textwidth}{!}{
            \begin{threeparttable}
            \begin{tabular}{|cllll|cllll|}
            \hline
            \multicolumn{5}{|c|}{\cellcolor[HTML]{D9EAD3}Integrante} & \multicolumn{5}{c|}{\cellcolor[HTML]{D9EAD3}Rol}                                          \\ \hline
            \multicolumn{5}{|c|}{Andrés David Beltrán Rojas}        & \multicolumn{5}{c|}{Gerente general, líder de la área de ingeniería}                      \\ \hline
            \multicolumn{5}{|c|}{Luisa María Lugo Flórez}   & \multicolumn{5}{c|}{Directora del departamento de investigación, desarrollo e innovación} \\ \hline
            \end{tabular}%
            
            \begin{tablenotes}[para,flushleft]
                \vspace{2mm}
                \textit Nota. Fuente : Autores.
            \end{tablenotes}
            \end{threeparttable} 
            }    
    \end{adjustbox}
    }