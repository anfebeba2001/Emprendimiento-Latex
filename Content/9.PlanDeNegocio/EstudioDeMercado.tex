\subsection{Estudio de mercado}
\subsection*{Identificación del producto}
El producto es una plataforma enfocada en la comodidad y seguridad de los nomadas digitales en Bogotá que buscan un lugar seguro y de calidad para vivir y trabajar. La plataforma ofrece información sobre la seguridad de las zonas de la ciudad, recomendaciones de lugares que se acoplan al perfil del usuario basandonos en analisis de gustos y experiencias, así como una comunidad de apoyo para los nómadas digitales.

La plataforma ofrece las siguientes funcionalidades:
\begin{itemize}
    \item Registro de usuario: Los usuarios pueden registrarse y acceder con su usuario y contraseña para disfrutar de todas las funciones que el sistema brinda.
    \item Evaluación integral de seguridad: Analiza en tiempo real factores como niveles de criminalidad, iluminación, afluencia de personas y accesibilidad, combinando datos oficiales con inteligencia artificial
    \item Recomendaciones personalizadas: Basadas en el perfil del usuario, preferencias y experiencias previas, la plataforma sugiere zonas y lugares que se alinean con sus necesidades.
    \item Ecosistema colaborativo: Una comunidad de apoyo para los nómadas digitales,compartiendo espacios nomad-friendly(Ej: zonas de coworking 24/7, lugares con buena conexión a internet, etc).
\end{itemize}

\subsection*{Demanda}
El proyecto satisface la creciente demanda de nómadas digitales y profesionales remotos en Bogotá que buscan un lugar seguro y de calidad para vivir y trabajar, valorando la seguridad y la comodidad en su entorno.Además de empresas globales que envian talento remoto a Bogotá y requieren garantizar la seguridad de sus empleados.

Se ha demostrado un crecimiento de nómadas digitales en Colombia en 2025, Bogotá, Medellín y Santa Marta se consolidaron como destinos clave para nómadas digitales, impulsado por la visa creada en 2022 que permite una estadía de hasta 2 años con un ingreso mínimo mensual de USD 900.\cite{crecimiento_nomadas_digitales}

Además se tiene que el volumen de búsqueda e interés en 2024 hubo unas 2.460 búsquedas mensuales en Google relacionadas con la visa para nómadas digitales en Colombia, representando el 26\% del interés desde Bogotá.\cite{volumen_nomadas_digitales}



\subsection*{Oferta}
Comprendiendo la demanda mencionada, se debe analizar la oferta existente en el mercado para identificar las ventajas competitivas de la plataforma. Estas ventajas se traducen en un valor agregado significativo para los usuarios, como:
\begin{itemize}
    \item Una experiencia de usuario personalizada que integra la seguridad y la comodidad en un solo lugar, facilitando la vida diaria de los nómadas digitales.
    
    \item Una interfaz que no solo permite a los usuarios gestionar su seguridad, sino también descubrir y conectarse con otros nómadas digitales, proporcionando una solución integral para la vida nómada.
    
    \item La función de recomendaciones personalizadas que ofrece sugerencias basadas en las preferencias y el historial de actividad de los usuarios, incentivando la participación en la comunidad local.
    \item Existen apps como Citizen (EEUU) y WalkSafe (Reino Unido) que proporcionan alertas de seguridad en tiempo real, lo cual demuestra demanda por herramientas de ese tipo.
    \item Se ha demostrado la necesidad de una plataforma que supla las necesidades de los nómadas digitales en Bogotá ya que se tiene que Bogotá presenta una combinación de costos accesibles (alrededor de USD 1.265/mes), alta velocidad de internet ($\approx$135 Mbps), conectividad móvil y una comunidad nómada creciente.\cite{entorno_nomadas_digitales}
\end{itemize}

Se puede observar que las apps actuales suelen enfocarse en espacios de trabajo o alertas de seguridad de forma aislada, mientras nuestra plataforma integra ambos aspectos más la personalización por perfil del usuario.Con un enfoque fuerte en la comunidad que no solo informa sino permite interacción, creación de comunidad y soporte entre nómadas, llenando un vacío en las ofertas del mercado.
Con la funcionalidad de combinar datos de seguridad, cultura, clima y eventos se garantiza una experiencia segura y actualizada, diferenciándose de soluciones que no consideran el contexto urbano en tiempo real.

\subsection*{Plan de marketing}
El marketing estratégico es vital para conectar con nómadas digitales y empresas en Bogotá, posicionándose como la plataforma líder en seguridad verificada. Estas acciones no solo generarán reconocimiento local, sino que sentarán las bases para su expansión internacional, construyendo una base de usuarios leales y una marca confiable en el ecosistema nómada.
\textbf{Objetivos}
\begin{itemize}
    \item Lanzar la plataforma destacando su propuesta única de seguridad inteligente para nómadas digitales en Bogotá, con recomendaciones validadas y actualizadas en tiempo real.
    \item Lograr una adopción rápida entre nómadas y empresas mediante campañas digitales segmentadas y alianzas con espacios nomad-friendly.
    \item Fidelizar usuarios con alertas personalizadas, contenido útil sobre seguridad urbana y una comunidad colaborativa dentro de la app.
    \item Posicionarse como referente en seguridad para nómadas digitales en Colombia, con proyección a mercados como Medellín.
    \item Maximizar ingresos mediante un modelo freemium, suscripciones premium, alianzas con coworkings y hostales certificados.

\end{itemize}

\textbf{Táctica}
El proyecto implementará un enfoque de inbound marketing,siendo una metodología centrada en atraer, interactuar y deleitar al público objetivo mediante contenido valioso y no intrusivo. Este enfoque es ideal para construir confianza en un mercado donde la seguridad es prioritaria.

\begin{itemize}
    \item \textbf{Enfoque en Contenido de Valor:}Se basa en la creación y distribución de contenido útil (blogs, guías, webinars) diseñado para resolver problemas específicos del público objetivo, sin vender directamente.
    \item \textbf{Atracción Orgánica (No Intrusiva):}Utiliza estrategias como SEO, redes sociales y email marketing para atraer tráfico cualificado que busca activamente soluciones.
    \item \textbf{Segmentación por Buyer Personas:}El contenido se personaliza según las necesidades, etapas del buyer's journey y características demográficas/conductuales de cada segmento.
    \item \textbf{Construcción de Relaciones a Largo Plazo:}Prioriza la fidelización mediante comunidades, soporte continuo y experiencias personalizadas, convirtiendo usuarios en promotores de la marca.
\end{itemize}
El objetivo principal de la metodología inbound marketing es atraer, convertir, cerrar y deleitar a los clientes potenciales mediante estrategias no intrusivas, centradas en ofrecer valor, resolver problemas y construir relaciones a largo plazo. Se diferencia del marketing tradicional al priorizar la educación sobre la venta directa, guiando al público a través de un proceso natural de decisión.

Como se observa en la figura \ref{inboundMarkting-1} se muestran las etapas del inbound marketing que se implementarán en el proyecto, para asegurar el correcto desarrollo de la estrategia de marketing y la satisfacción del cliente.



\vspace{2mm}
        \begin{minipage}{0.9\textwidth}
        \centering
        \captionof{figure}[{Fases Inbound Marketing}]{ Fases Inbound Marketing  }
        \label{inboundMarkting-1}
         \includegraphics[width=0.8\textwidth]{Content/Images/Metodología Inbound-1.jpeg}
        \footnote{Nota. \textup{Fuente: Inbound Marketing, la guía completa \cite{InboundMarketing}}}
\end{minipage}

\begin{itemize}
    \item \textbf{Atraer:}Esta primera etapa se centra en atraer nómadas digitales que busquen una solución a sus problemas encontrando lugares seguros y de calidad para vivir y trabajar. Utilizando estrategias de marketing de contenido, como la creación de contenido útil (guías, blogs, infografías, etc.) orientado a resolver sus necesidades y posicionar la plataforma como referente en seguridad y comodidad para nómadas digitales en Bogotá.
    \item \textbf{Convertir:}Luego de atraer a los nómadas digitales, el objetivo es convertir a los usuarios en leads, es decir, en contactos interesados en la plataforma. Esto ofreciendo recursos descargables como mapas, rutas y demás que puedan ser de interés para el usuario, a cambio de su información de contacto como email, utilizando formularios optimizados.
    \item \textbf{Cerrar:}Una vez que se han convertido en leads, el siguiente paso es cerrar la venta. Esto se logra mediante el uso de herramientas de automatización de marketing y email marketing, Automatizando emails personalizados (ej.: serie de onboarding con tips de seguridad) y contenido relevante para guiar a los leads hacia la decisión de compra. Se utilizarán técnicas de lead scoring para identificar a los leads más calificados y priorizar su seguimiento.
    \item \textbf{Deleitar:}Por ultimo se busca deleitar a los usuarios, convirtiéndolos en promotores de la plataforma. Esto se logra mediante la creación de una comunidad activa y participativa, donde los usuarios puedan compartir sus experiencias, recomendaciones y consejos sobre seguridad y bienestar en Bogotá. Se fomentará la interacción a través de foros y redes sociales, creando un sentido de pertenencia y lealtad hacia la marca.
\end{itemize}

Permitiendo vincularnos mas profundamente con los nómadas digitales y crear una comunidad sólida que respalde la plataforma. Nuestro proyecto les brindará valor y soluciones a sus problemas, generando confianza y lealtad hacia la marca,aumentando el alcance de la misma a base de su fidelidad y confianza.

\subsection*{Precio}
El precio de la plataforma se ha establecido teniendo en cuenta el análisis de la competencia y el valor agregado que ofrece. Se ha realizado un estudio de mercado para determinar el rango de precios de servicios similares en la región, considerando factores como la calidad del servicio, la personalización y la seguridad.
El precio de la plataforma se ha fijado en un rango competitivo. Se espera que el precio sea atractivo para los nómadas digitales y empresas que buscan soluciones de seguridad y bienestar personalizadas, el precio se ha calculado en base a los siguientes factores:

\begin{itemize}
    \item Valor de Seguridad Verificada:Crear un valor agregado único que justifique el precio premium, destacando que el proyecto ofrece datos de seguridad validados en tiempo real por expertos locales, no solo información crowdsourced como la competencia.
    \item Modelo de Monetización Flexible:Asegurar la sostenibilidad financiera mediante un modelo escalable que combine opciones freemium (acceso básico gratuito) con suscripciones premium ($5-$15 USD/mes) para funcionalidades avanzadas como alertas personalizadas.
    \item Benchmarking Competitivo: Analizar exhaustivamente los precios y características de alternativas como NomadList y SafeCity para posicionar a Triavel como la opción más equilibrada entre costo y valor real en seguridad verificada.
    \item Demanda y Disposición a Pagar:Evaluar cuidadosamente la disposición a pagar del mercado objetivo mediante estudios cuantitativos y cualitativos, considerando tanto nómadas individuales como empresas con empleados remotos.
    \item Fidelización mediante Experiencia:Desarrollar programas de lealtad que transformen usuarios ocasionales en clientes recurrentes, ofreciendo beneficios exclusivos como validaciones prioritarias o acceso a eventos comunitarios.
    \item Soporte Técnico como Diferenciador:Incluir en la estructura de precios el costo de ofrecer un soporte técnico excepcional y personalizado, convirtiéndolo en un argumento de venta único frente a competidores con atención automatizada.
\end{itemize}

\begin{adjustbox}{
            center,
            caption=[{Precio en el mercado actual}]{\centering Precio en el mercado actual (Valores en pesos colombianos COP). },
            label={Precio},
            nofloat=table, vspace={20px}}
            {
       \begin{threeparttable}
           \begin{tabular}{|p{11cm}|p{10cm}p{2cm}|}
                \hline
                \rowcolor[HTML]{D9EAD3} 
                \cellcolor[HTML]{D9EAD3}                              & \multicolumn{2}{c|}{\cellcolor[HTML]{D9EAD3}Precio}            \\ \cline{2-3} 
                \rowcolor[HTML]{D9EAD3} 
                 \multirow{-2}{*}{\cellcolor[HTML]{D9EAD3}Suscripción} & \multicolumn{1}{l|}{\cellcolor[HTML]{D9EAD3}Mínimo} & Máximo   \\ \hline
                                \multicolumn{1}{|l|}{Mensualidad}                     & \multicolumn{1}{l|}{\$30000}                       & \$3600000 \\ \hline
                                \multicolumn{1}{|l|}{Anualidad}                       & \multicolumn{1}{l|}{\$360000}                      & \$3600000 \\ \hline      \end{tabular}
            \begin{tablenotes}[para,flushleft]
                \vspace{2mm}
               \textit Nota. Fuente: Autores.
            \end{tablenotes}
            
        \end{threeparttable}
    }

\end{adjustbox}

El precio de la plataforma se ha fijado en un rango competitivo, con una mensualidad que oscila entre \$50.000 y \$300.000 COP, dependiendo de las funcionalidades y servicios adicionales que se ofrezcan. Este rango se ha establecido tras un exhaustivo análisis de la competencia y la identificación de las necesidades del mercado objetivo.
Se tendrá en cuenta la posibilidad de ofrecer diferentes planes de suscripción, que incluyan desde un acceso básico gratuito hasta opciones premium con funcionalidades avanzadas, como alertas personalizadas y validaciones prioritarias. Esto permitirá captar una amplia gama de usuarios, desde nómadas digitales individuales hasta empresas de turismo que buscan expandir sus operaciones en Bogotá, dichas opciones de suscripción se detallan a continuación:

\begin{adjustbox}{
            center,
            caption=[{Precios de suscripción.}]{Precios de suscripción mensual y anual },
            label={PrecioPeriodos},
            nofloat=table, vspace={20px}}
            {
            \begin{threeparttable}
                \begin{tabular}{|p{7cm}|p{4cm}|p{4cm}|}
                    \hline
                    \rowcolor[HTML]{D9EAD3}
                    Suscripción & Mensualidad & Anualidad \\ \hline
                    Plan gratuito & \$0 & \$0 \\ \hline
                    Plan personal premium & \$30.000 & \$360.000 \\ \hline
                    Plan empresarial & \$240.000 & \$2.880.000 \\ \hline
                    Plan empresarial premium & \$300.000 & \$3.600.000 \\ \hline
                \end{tabular}%
                \begin{tablenotes}[para,flushleft]
                    \vspace{2mm}
                    \textit Nota. Fuente: Autores.
                \end{tablenotes}
            \end{threeparttable}
    }
\end{adjustbox}